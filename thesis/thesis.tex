\documentclass[twoside,english]{uiofysmaster}

\usepackage{pdfpages}
\usepackage{cite}
\usepackage{epsfig}
\usepackage{caption}
\usepackage{subcaption}
\usepackage[normalem]{ulem}
\usepackage{textcomp}
\usepackage{varioref}
\usepackage{slashed}
\usepackage{csquotes}
\usepackage{amssymb}

%\bibliography{references}

\author{J\o rgen Eriksson Midtb\o}
\title{Noe}
\date{June 2015}



\begin{document}

\pagenumbering{roman}
\includepdf{front-page.pdf}
\cleardoublepage

\begin{abstract}
This is an abstract text.
\end{abstract}

%\begin{dedication}
%  Til Nina Eriksson
%  \\\vspace{12pt}
%  This is a dedication
%\end{dedication}

% \begin{acknowledgements}
% Takk til Are Raklev for god rettleiing, til Lars Dal og Anders Kvellestad for god hjelp, og til Anders Lauvland og Anders Hafreager for gode diskusjonar. Takk til Janne for at du er du. Takk til pappa. Takk også til Bryan Webber, Mona Semb, Tor Gjerrestad, Arild [], 
% \end{acknowledgements}

\tableofcontents
\listoffigures
\listoftables



%%%%%%%%%%%%%%%%%%%%%%%%%%%%%%%%%%%%%%%%%%%%%%%%%%%%%%%%%%%
\chapter{The Standard Model of Particle Physics}%%%%%%%%%%%%%%%%%%%%%%%%%%%%%%%%%%%%
%%%%%%%%%%%%%%%%%%%%%%%%%%%%%%%%%%%%%%%%%%%%%%%%%%%%%%%%%%%
\label{ch:SM_intro}
The Standard Model of particle physics has been hugely successful in explaining what our universe consists of at the smallest length scales, and how these constituents interact with each other. It recieved a final, spectacular confirmation in 2012, when a Higgs boson consistent with the predictions of the Standard Model was discovered by the CMS and ATLAS experiments at CERN. It is well known, however, that the Standard Model is incomplete as a description of our universe, for instance since it gives no explanation for dark matter. There are also more technincal problems with the Standard Model, such as the hierarchy problem of the Higgs boson mass loop corrections and the arbitrariness of the model parameters.

The present chapter gives an introduction to the principles that underlie the construction of the Standard Model, and outlines the derivation of the model.

\section{Symmetries and conservation laws}
Symmetries are manifest in many physical systems. For instance, the special theory of relativity is symmetric under boosts and rotations, as well as translations in space and time. There is a deep relationship between symmetries and the conservation of physical quantities. This result is known as Noether's theorem, and was proven by Emmy Noether in 1915. It states that {\it every differentiable symmetry of the action of a physical system has a corresponding conservation law}. In the example of special relativity, the symmetries under translations in time and space correspond to conservation of energy and momentum.%, and the symmetry under rotation corresponds to conservation of angular momentum. \marginpar{What is the conserved quantity under boosts? Is it that ljyubliajsnski polarization vector? Check.}

\subsection{Description by groups}
It is often convenient to describe the symmetries of physical systems in the language of group theory. A group is a set of objects which are closed under some binary operation -- meaning that any combination of two group elements yield another group element. The set of all Lorentz boosts and rotations in special relativity form a group, called the Lorentz group, and together with all spatial translations they form the Poincar\'{e} group. 

The experimental fact that there exist a number of conserved quantities in particle physical systems -- examples include energy and momentum, but also electrical and colour charge, among others -- can be used to construct a theory of particle interactions, by finding the symmetries, and the symmetry groups, that correspond to these quantities and demanding that the theory be symmetric under their action.

\section{The Standard Model of Particle Physics}

\subsection{Phenomenology of the Standard Model}
The Standard Model consists of 12 fermions with corresponding antifermions, a number of vector gauge bosons and one scalar boson. The gauge bosons mediate interactions between the particles. There are three fundamental interactions in the Standard Model: The electromagnetic interaction, the weak interaction and the strong interaction. Not all particles couple to each other with all of the interactions. 

The fermions are divided into two groups, the quarks and leptons. There are six different {\it flavours} of quarks, called up, down, strange, charm, bottom and top, in order of increasing mass. They are subdivided into three generations of pairs, up/down, charm/strange and top/bottom. The up, charm and top quarks carry quanta of +2/3 of the fundamental electrical charge $e$, while the down, strange and bottom quarks carry -1/3 $e$. There are also six leptons, of which three are charged. They are called electron, muon and tau. They belong in each their own generation, together with their neutral counterpart, the electron neutrino, muon neutrino and tau neutrino, respectively. 

The vector gauge bosons consist of the photon, the $Z$ and $W$ bosons and the gluon. The photon is the mediator of electromagnetic interactions, the $Z$ and $W$ mediate the weak interaction and the gluon mediates the strong interaction. The photon, $Z$ boson and gluon are all neutral, and they are their own antiparticles. The photon and gluon are massless, while the $W$ and $Z$ bosons are quite heavy. The $W$ carries one elementary unit of electric charge, and is thus distinct from its antiparticle, with a difference in sign for the particle and antiparticle states. The scalar boson of the Standard Model is the Higgs boson, which is responsible for giving particles their observed mass through the Higgs mechanism. It is electrically neutral and very massive.

Among the fermions, only the quarks couple to the strong interaction. All the fermions couple with the weak interaction, while only the electrically charged particles couple electromagnetically -- {\it i.e.}\ all except the neutrinos. They couple to the Higgs field according to their mass, so that for instance the top quark, which is the heaviest Standard Model particle, couples the strongest, and the neutrinos, being very light, hardly couple at all. A schematic overview of the particles in the Standard Model are shown in fig. \ref{fig:SM_particles}.
\begin{figure}[hbt]
	\centering
	\includegraphics[width=0.5\textwidth]{figures/susyintro/Standard_Model_of_Elementary_Particles.pdf}
	\caption{An overview of the particles of the Standard Model and their interactions, from \cite{Wikimedia_SM_particles}.}
	\label{fig:SM_particles}
\end{figure}

The flavours of the quarks and leptons are conserved in the electromagnetic and strong interactions. For instance, a top quark cannot change into a charm or up quark by emission of a photon or gluon. The weak interaction enables the top quark to change into a bottom quark, or a tau lepton to change into a tau neutrino, through the emission of a charged $W$ boson. This would still seem to conserve the {\it generation} of quark or lepton, but breaking of generation is also made possible through the mechanism of {\it generation mixing}, quantified by the Cabibbo-Kobayashi-Maskawa (CKM) matrix for the case of quarks and the Pontecorvo-Maki-Nakagawa-Sakata (PMNS) matrix for the leptons. The PMNS mixing also explains the observed phenomenon of {\it neutrino oscillations}.

\subsection{Constructing the Lagrangian of the Standard Model}

The Standard Model is a quantum field theoretic model, and may be stated in terms of a Lagrangian density function $\mathcal{L}$. The guiding principle for constructing the Lagrangian is {\it gauge invariance}. Gauge degrees of freedom are physical degrees of freedom which are ``superflous'', in the sense that they do not have any observable consequences. An example is Maxwell's theory of electromagnetism, where the electromagnetic vector potential $A^\mu$ is undetermined up to the addition of a total derivative term $\partial^\mu \phi$. The gauge freedom is exploited by requiring that the Lagrangian, which determines the physical dynamics, does not change when the gauge degrees of freedom are varied -- that it is gauge invariant. This invariance is related to conservation of physical quantities by Noether's theorem.

\subsection{The Lie groups of the Standard Model}

The Standard Model is based on gauge invariance under three Lie groups of quadratic matrices, the infamous $U(1)_Y\times SU(2)_L\times SU(3)_C$. The number in the parenthesis gives the matrix dimension of the group. The symbols $S$ and $U$ stand for {\it special} and {\it unitary}, respectively. Unitary means that the matrices are unitary, and special means they have determinant 1. The groups are Lie groups, which means that they are continuous, and thus that any transformation of a group element may be constructed from infinitesimal transformations. The group elements, and the objects on which the group acts, may be given in several {\it representations}. In the case of matrix groups this means matrices and vectors of different dimension. For an $SU(n)$ group, the two most important representations are the {\it fundamental} representation, where the vectors have dimension $n$, and the {\it adjoint} representation, where the vectors have dimension $n^2-1$. In the Standard Model, the fermions transform in the fundamental representation, while the gauge bosons transform in the adjoint representation.

An element of a Lie group may generally be written as\footnote{Here and in the following, repeated indices are summed over.} 
\begin{align}
	e^{i\alpha_a T_a},
\end{align}
where $T_a$ are the $n^2-1$ generators of the Lie algebra of the group. The generators themselves are in the fundamental representation represented by traceless complex $n\times n$ matrices -- for $SU(2)$, these are the Pauli matrices $\sigma_i$, and for $SU(3)$ they are the Gell-Mann matrices $\lambda_i$ -- and in the adjoint representation by the {\it structure coefficients} $f_{abc}$ as $(T_a)_{bc} = f_{abc}$. The structure coefficients are determined from the Lie algebra as
\begin{align}
	[T_a, T_b] = i f_{abc}T_c.
\end{align}

\subsection{Constructing a gauge theory}
The particle content of the Standard Model is input into the Lagrangian by inserting fermionic fields, {\it i.e.}\ Dirac spinor fields, and imposing the desired gauge invariance on these fields. The basic Dirac term, called the Dirac bilinear, for some spinor field $\psi$, is\footnote{We will, for what follows, set $\hbar = c = 1$.} 
\begin{align}
	\bar \psi (i\gamma^\mu \partial_\mu - m) \psi = \bar \psi (i\slashed\partial - m)\psi, \label{eq:diracbilinear}
\end{align}
where $\gamma_\mu$ are the Dirac matrices, $m$ is the mass of the spinor field, and $\bar\psi \equiv \psi^\dag \gamma_0$. Next, we impose gauge invariance. The group transformation of an $SU(n)$ group may be written in the fundamental representation as
\begin{align}
	G(x) = e^{(ig\alpha_a(x)T^a)},
\end{align}
where $\alpha(x)$ are $n$ arbitrary real differentiable functions and $T^a$ are the generators of $SU(n)$ in the fundamental representation. We assume that the Lagrangian consists of $n$ Dirac bilinear terms with fields $\psi_i$, and that they are put into an $n$-dimensional multiplet $\Psi = (\psi_1, \psi_2, ..., \psi_n)^T$ such that the basic Dirac Lagrangian reads
\begin{align}
	\mathcal{L}_0 = \bar\Psi(i\slashed\partial - m)\Psi
\end{align}
where we assume for simplicity that all fields have the same mass $m$.\footnote{This assumption is often wrong in the case of the Standard Model, but finds its solution in the Higgs mechanism.} The group transformations of the multiplet and its adjoint are then 
\begin{align}
	\Psi(x) &\overset{G}{\to} e^{(ig\alpha_a(x)T^a)} \Psi(x),\\
	\bar\Psi(x) &\overset{G}{\to} \bar\Psi(x) e^{(-ig\alpha_a(x)T^a)}.\nonumber
\end{align}
If we apply these transformations to the basic Lagrangian, it becomes
\begin{align}
	\mathcal{L}_0 = &\bar\Psi(x)(i\slashed\partial - m)\Psi(x)\nonumber\\
	\overset{G}{\to} &\bar\Psi(x) e^{(-ig\alpha_a(x)T^a)}(i\slashed\partial - m)e^{(ig\alpha_a(x)T^a)} \Psi(x)\\
	=	&\bar\Psi(x)(i\slashed\partial - m)\Psi(x) - g\bar\Psi(x) T^a\slashed\partial \alpha^a(x) \Psi(x).\nonumber
\end{align}
Thus, the basic Dirac Lagrangian is not gauge invariant, since we have picked up an additional term. Gauge invariance may be achieved by adding a term of the form 
\begin{align}
	-\bar\Psi(x) ig\gamma^\mu \alpha_a(x)A^a_\mu(x) \Psi(x)\label{eq:covariantderivativeterm}
\end{align}
to the Lagrangian, where $A^a_\mu(x)$ is some new field, which we require to transform under $G$ as
\begin{align}
	A^a_\mu(x) \overset{G}{\to} A^a_\mu(x) + \partial_\mu \alpha^a(x).
\end{align}
If we apply $G$ to the sum of the Dirac bilinear with this new term, it is invariant:
\begin{align}
	&\bar\Psi(x)(i\slashed\partial - m)\Psi(x) - \bar\Psi(x) ig\gamma^\mu \alpha_a(x)A^a_\mu(x) \Psi(x)\nonumber\\
	\overset{G}{\to} &\bar\Psi(x)(i\slashed\partial - m)\Psi(x) - g\bar\Psi(x) T^a\slashed\partial \alpha^a(x) \Psi(x)\\
	 &- \bar\Psi(x) ig\gamma^\mu \alpha_a(x)A^a_\mu(x) \Psi(x) +  g\bar\Psi(x) T^a\slashed\partial \alpha^a(x) \Psi(x)\nonumber\\
	 = &\bar\Psi(x)(i\slashed\partial - m)\Psi(x) - \bar\Psi(x) ig\gamma^\mu \alpha_a(x)A^a_\mu(x) \Psi(x).\nonumber
\end{align}
The term from eq.\ \eqref{eq:covariantderivativeterm} is usually included by replacing $\partial_\mu$ with the covariant derivative $D_\mu = \partial_\mu + igT_a A^a_\mu$. The fields $A^a_\mu$ are called gauge boson fields, and are responsible for mediating interactions between the Dirac fermions. The gauge boson fields must also have their own free-field term, called the field strength, which is given from the Proca Lagrangian for spin-1 fields as 
\begin{align}
	-\frac{1}{4} F_{a,\mu\nu} F^{a,\mu\nu},
\end{align}
where
\begin{align}
	F_{a,\mu\nu} \equiv [D_{a,\mu}, D_{a,\nu}] = \partial^\mu A_a^\nu - \partial^\nu A_a^\mu + g f_{abc} A^{b,\mu}(x)A^{c,\nu}(x),
\end{align}
where $f_{abc}$ are the structure coefficients of $SU(n)$.

With this, the total gauge invariant Lagrangian consists of $n$ fermion fields and $n^2-1$ gauge boson fields, and reads
\begin{align}
	\mathcal{L} = \bar\Psi(i\slashed D - m)\Psi - \frac{1}{4} F_{a,\mu\nu} F^{a,\mu\nu}.
\end{align}
The covariant derivative gives rise to terms coupling the fermion and gauge fields together. In the case of $n=1$, the gauge group is the $U(1)$ group which describes the theory of quantum electrodynamics, the simplest realistic gauge theory. For $U(1)$, the structure coefficients vanish, since there is only a single gauge field\footnote{This contradicts the claim that there are $n^2-1$ gauge fields. The reason is that $U(1)$ is not an $SU(n)$ group, but the above derivation works for $U(1)$ as well.}, making the Lagrangian particularily simple. In QED, there are no gauge boson self-interactions. For $n>1$, the structure coefficients do not vanish, and this gives rise to terms in the field strength term $-1/4 F^2$ coupling the gauge bosons among themselves. These interactions are of great importance in the theories of weak and strong interactions.




% \marginpar{Do this for $SU(n)$ instead!} The simplest example is a $U(1)$ symmetry, where the group transformations are simply local phase transformations 
% \begin{align}
% 	G(x) = \exp(iq\alpha(x)).
% \end{align}
% If we apply this transformation to both spinor fields in eq.\ \eqref{eq:diracbilinear}, we get
% \begin{align}
% 	\bar \psi (i\slashed\partial - m)\psi \overset{G}{\to} \bar \psi (i\slashed\partial - m)\psi - q(\partial_\mu \alpha(x))\bar\psi \gamma^\mu \psi.
% \end{align}
% Thus the Dirac bilinear is not gauge invariant, since we have picked up an additional term. Gauge invariance may be achieved by adding a term of the form
% \begin{align}
% 	-(q\bar\psi \gamma^\mu \psi)A_\mu,\label{eq:covariantderivativeterm}
% \end{align}
% to the Lagrangian, where $A_\mu = A_\mu(x)$ is some new field, which we require to transform under $G$ as
% \begin{align}
% 	A_\mu \overset{G}{\to} A_\mu + \partial_\mu \alpha(x).
% \end{align}
% If we apply $G$ to the sum of the Dirac bilinear with this new term, it is invariant:
% \begin{align}
% 	&\bar \psi (i\slashed\partial - m)\psi -(q\bar\psi \gamma^\mu \psi)A_\mu\\
% 	\overset{G}{\to}&  \bar \psi (i\slashed\partial - m)\psi  - q(\partial_\mu \alpha(x))\bar\psi \gamma^\mu \psi - (q\bar\psi \gamma^\mu \psi)A_\mu + q(\partial_\mu \alpha(x))\bar\psi \gamma^\mu \psi\\
% 	=& \bar \psi (i\slashed\partial - m)\psi -(q\bar\psi \gamma^\mu \psi)A_\mu.
% \end{align}
% The term from eq.\ \eqref{eq:covariantderivativeterm} is usually included by replacing $\partial^\mu$ with the covariant derivative $D^\mu = \partial^\mu + iqA^\mu$. The field $A_\mu$ is called a gauge boson field, and is responsible for mediating interactions between the Dirac fermions. The gauge boson field must also have its own free-field term, called the field strength, which is
% \begin{align}
% 	\frac{1}{4} F^{\mu\nu}F_{\mu\nu},
% \end{align}
% where $F^{\mu\nu} \equiv \partial^\mu A^\nu - \partial^\nu A^\mu.$ The total Lagrangian for a single fermion subject to a $U(1)$ interaction is then
% \begin{align}
% 	\mathcal{L} = \bar \psi (i\slashed D - m)\psi - \frac{1}{4} F^{\mu\nu}F_{\mu\nu}.
% \end{align}

% In the case of $SU(2)$ and $SU(3)$, some additional complications arise. Firstly, the fermion spinors must be in doublets and triplets, respectively, to be able to transform under the fundamental representation of the groups. Secondly, they contain more than one gauge field. Thirdly, the Lie algebra of the groups is non-trivial, {\it i.e.}\ the structure coefficients $f_{abc}$ are nonzero. This complicates the definition of the field strength tensor $F^{\mu\nu}$, which is generally defined as
% \begin{align}
% 	F^{\mu\nu} = \frac{1}{g}[D^\mu, D^\nu],
% \end{align}
% where $D^\mu$ again is the covariant derivative, which generally is defined as
% \begin{align}
% 	D^\mu = \partial^\mu + ig T_a A^a_\mu,
% \end{align}
% where $T_a$ are the generators of the group. This was implicit in the $U(1)$ example because $U(1)$ has only a single generator which is proportional to the identity. In the case of vanishing structure coefficients, the expression for the $U(1)$ field strength tensor $F^{\mu\nu} \equiv \partial^\mu A^\nu - \partial^\nu A^\mu$ is recovered. The nonzero structure coefficients will enter into the field strength tensor through the commutator, and give rise to terms in the contracted field strength $F^{\mu\nu}F_{\mu\nu}$ which couple the gauge fields with each other. This gives rise to the vector boson self-interactions that are present in the weak and strong interaction, but absent in the electromagnetic interaction because it has a $U(1)$ symmetry.

\subsection{Singlets, doublets and triplets}

Not all the fields are subject to all the different interactions. If a field couples through a certain interaction, it is said to be {\it charged} under the transformations corresponding to that interaction. In the electromagnetic $U(1)$ case discussed earlier, this charge is the electrical charge. A specific amount $q$ of electrical charge is assigned to every field, and enters into the group transformations as $G(x) = \exp(iq\alpha(x))$. Thus, for $q=0$, the transformation is the identity and has no effect. Analogous charges are associated with the $U(1)_Y$, $SU(2)_L$ and $SU(3)_C$ groups. They are called hypercharge, isospin and colour charge, respectively.

In the case of $SU(2)$ and $SU(3)$, the fields have to be put into vectors in order to be acted upon by the transformations. As previously mentioned, the fermionic fields transform in the fundamental representation, meaning that the dimension of the vectors is 2 and 3, respectively. These types of vectors are referred to as $SU(2)$ {\it doublets} and $SU(3)$ {\it triplets}.

A Dirac field can be written as the sum of a left-chiral and a right-chiral part, defined by the projection operators 
\begin{align}
	P_{R/L} = \frac{1\pm \gamma^5}{2},
\end{align}
where $\gamma^5 \equiv i\gamma^0\gamma^1\gamma^2\gamma^3$. Given a Dirac field $\psi$, we may write
\begin{align}
	\psi = \left( \frac{1 + \gamma^5}{2} + \frac{1 - \gamma^5}{2}\right)\psi = P_R \psi + P_L \psi \equiv \psi_R + \psi_L.
\end{align}
In the case of $SU(2)_L$, only the {\it left chiral} part of the fields are charged under the symmetry. For instance, the left-chiral parts of the quark fields are put in doublets, {\it e.g.}\
\begin{align}
	q_L = \begin{pmatrix}
		u_L \\ d_L
	\end{pmatrix},
\end{align}
for the up- and down-quarks, while the right-handed parts of each of the quark fields are put in two separate singlets $u_R$ and $d_R$, upon which the $SU(2)_L$ transformation does not act. This has the consequence that the $SU(2)_L$ interaction is left-chiral -- it only couples to left-handed fields. Due to the spontaneous symmetry breaking of the $U(1)_Y\times SU(2)_L$ symmetry, the chirality is not exact in the resulting weak interactions, but it is still an important feature of the Standard Model.

The $SU(3)_C$ symmetry is the symmetry of the strong nuclear force, and among the fermions, only the quarks are charged under it. The quarks transform under $SU(3)$ in triplets -- one for each quark flavour -- where the components of the triplet are discriminated by differing {\it colour}, red, green or blue. 

\subsubsection{The gauge bosons and the adjoint representation}

While the fermions transform under the groups in the fundamental representation, which has dimension $n$ for $SU(n)$, the gauge vector boson fields transform in the adjoint representation, which has dimension $n^2-1$. This number then determines the number of different gauge bosons for each group: $U(1)_Y$ has a single gauge boson field labeled $B^\mu$; $SU(2)_L$ has three, labeled $W^\mu_{1,2,3}$; and $SU(3)_C$ has eight different gauge boson fields, labeled $A^\mu_a$ for $a = 1,...,8$. The $SU(3)$ bosons are called {\it gluons}. The $U(1)_Y$ and $SU(2)_L$ bosons are not the ones that we observe -- the physical gauge boson eigenstates are linear combinations of them, mixed together by the spontaneous symmetry breaking of the Higgs mechanism.

\subsection{The Higgs mechanism}
Gauge invariance forbids the inclusion of terms of the form $m^2A^\mu A_\mu$ into the Lagrangian, which are required in order to give vector bosons such as the $Z$ their observed mass. To include the terms, we must also include a new scalar field, called the Higgs field. The Higgs mechanism introduces the following terms into the Lagrangian:
\begin{align}
	\mathcal L \ni \partial^\mu \phi^*(x) \partial_\mu \phi(x) - \mu^2|\phi(x)|^2 - \lambda |\phi(x)|^4.
\end{align}
The last two terms comprise the Higgs {\it potential}. If $\mu^2$ is assumed to be negative and $\lambda$ positive, then the potential assumes the shape of a ``mexican hat'' as a function of $\phi$. This is shown in fig. \ref{fig:higgspot}.
\begin{figure}[hbt]
	\centering
	\includegraphics[width=0.5\textwidth]{figures/susyintro/higgspot_nature.jpg}
	\caption{The shape of the Higgs potential, stolen from Nature.com}
	\label{fig:higgspot}
\end{figure}
This potential has a circle of degenerate energy at the field value $|\phi| = (-\mu^2/2\lambda)^{1/2}$, or in terms of the actual complex field, $\phi = (-\mu^2/2\lambda)^{1/2}e^{i\theta}, \, \theta \in [0,2\pi)$. The mechanism of {\it spontaneous symmetry breaking} occurs when, as the energy decreases, the Higgs field falls to the bottom of the degenerate circle and is forced to {\it choose} a particular value of $\theta$. This breaks the gauge invariance of the Lagrangian under both $U(1)_Y$ and $SU(2)_L$, and causes the gauge bosons of the two groups to mix together: The $B^\mu$ field mixes with $W^\mu_3$ to form the fields $A^\mu$ and $Z^\mu$, by the assignments
\begin{align}
	W_3^\mu &= \cos\theta_W Z^\mu + \sin\theta_W A^\mu,\\
	B^\mu &= -\sin\theta_W Z^\mu + \cos\theta_W A^\mu,
\end{align}
where $\theta_W$ is the {\it Weinberg angle}. The two other $SU(2)_L$ gauge fields, $W_{1,2}^\mu$, mix togeher to form
\begin{align}
	W_\mu = \frac{1}{\sqrt{2}} \left[ W_1^\mu - iW_2^\mu \right]
\end{align}
and its adjoint $(W^\mu)^\dag$. The field $A^\mu$ is the photon field, $Ẑ^\mu$ is the Z boson field and $W^\mu$ is the $W^\pm$ bosons. The mixing restores the symmetry under a different $U(1)$ group, $U(1)_\mathrm{em}$, and the corresponding conserved charge is the electrical charge. The photon and Z boson are electrically netural, while the $W^\pm$ carry plus/minus one elementary unit of charge.

The Higgs mechanism also provides fermionic terms of the form $\psi_i y_{ij} \psi_j$. For $i=j$, these are mass terms of the form written in the Dirac bilinear, eq. \eqref{eq:diracbilinear}.

\subsection{The Feynman calculus, loop corrections}
\label{subsec:feynmancalculus}
Very few problems in the framework of the Standard Model can be solved exactly. Instead, calculations are done using perturbation theory to series expand the solution as a sum of increasingly complicated, but decreasingly important, contributions. Feynman invented a technique for visualizing these expansions using visual diagrams, known as Feynman diagrams. For instance, the problem of inelastic electron-positron scattering has as its leading contribution the diagram shown in fig. \ref{fig:feynmandiagram_a}. The next-to-leading order includes the diagrams in figs. \ref{fig:feynmandiagram} (b), (c) and (d).
\begin{figure}[htbp]
	\centering
	\begin{subfigure}[b]{0.45\textwidth}
		\centering
		\includegraphics[width=0.6\textwidth]{figures/susyintro/epscattering.pdf}
		\caption{ }
		\label{fig:feynmandiagram_a}
	\end{subfigure}
	\begin{subfigure}[b]{0.45\textwidth}
		\centering
		\includegraphics[width=0.6\textwidth]{figures/susyintro/epscattering_fermionloop.pdf}
		\caption{ }
		\label{fig:feynmandiagram_b}
	\end{subfigure}

	\begin{subfigure}[b]{0.45\textwidth}
		\centering
		\includegraphics[width=0.6\textwidth]{figures/susyintro/epscattering_fermioncorr.pdf}
		\caption{ }
		\label{fig:feynmandiagram_c}
	\end{subfigure}
	\begin{subfigure}[b]{0.45\textwidth}
		\centering
		\includegraphics[width=0.6\textwidth]{figures/susyintro/epscattering_vertexcorr.pdf}
		\caption{ }
		\label{fig:feynmandiagram_d}
	\end{subfigure}
	\caption{Feynman diagrams of contributions to inelastic $e^+ e^-$ scattering. Made using \cite{Binosi:2003yf}.}
	\label{fig:feynmandiagram}
\end{figure}
The Feynman calculus associates each diagram with a specific mathematical expression called the Feynman amplitude $\mathcal{M}$ for that diagram. When several diagrams are included, the total amplitude for the process is the sum of the amplitudes from each diagram. The physical quantities of interest are obtained by integrating the amplitude (or its absolute square) over all spin and momentum configurations of the system.

\subsection{Renormalization}
The subleading diagrams, like those in fig.  \ref{fig:feynmandiagram} (b--d), contain closed loops. These loops introduce extra momentum integrals into the calculations. Often, these integrals are divergent -- which is unacceptable from a physical viewpoint. The divergences can be understood and dealt with by using the techniques of {\it regularization} and {\it renormalization}. 
\subsubsection{Regularization}
Regularization is a means for parametrizing the divergence in terms of some small parameter $\epsilon$ which is zero in the physical limit. The most modern way to regularize a momentum integral is by using {\it dimensional regularization}: The original loop integral is an integral over four space-time dimensions, 
\begin{align}
	\int d^4 k.
\end{align}
Dimensional regularization makes the substitution $4 \to d = 4-\epsilon$, so the integral becomes
\begin{align}
	\int d^d k.
\end{align}
This is mathematically well-defined, and allows the divergences to be parametrized in terms of $\epsilon$.
\subsubsection{Renormalization}
When the divergence has been isolated and parametrized, it needs to be explained physically. This is done by the process of renormalizing the theory. For instance, in the case of the photon propagator in quantum electrodynamics, illustrated in fig. \ref{fig:feynmandiagram_b}, the regularized expression for the leading-order loop correction to the propagator is proportional to
\begin{align}
	\frac{2}{\epsilon} + \mathrm{const.}, \label{eq:divergent_loop_result}
\end{align}
which blows up as $\epsilon\to 0$. Renormalization is the claim that this infinity is a part of the {\it bare} physical constants which are present in the Lagrangian, in this case the electrical charge, whose bare value is denoted $e_0$. These bare parameters are not observable quantities, only parameters in the Lagrangian. What is observed is the {\it renormalized} charge $e = e_0 + \delta e$, where $\delta e$ is the infinite shift contributed from eq. \eqref{eq:divergent_loop_result}.

All the coupling constants of the Standard Model get renormalized. The renormalization introduces an {\it energy dependence} into the coupling constants, since the shift comes from loop corrections which depend on the energy of the process. For instance, the physical value of the electron charge in quantum electrodynamics, at some momentum $q$, is given as
\begin{align}
	e^2(q) = \frac{e_r^2}{1 - (e_r^2/6\pi^2)\log(q/M)},\label{eq:electron_charge_running}
\end{align}
where $e_r$ is some referance value for the charge, defined at the energy scale $M$. The fact that the coupling constants are not constant is referred to as the {\it running of the coupling constants}.

\subsubsection{The Callan-Symanzik equation}
The Callan-Symanzik, or {\it renormalization group} equation is the equation which describes the running of the coupling constants in a systematic way for any interaction in a quantum field theory. It is obtained by requiring that the Greens function for the interaction, $G$, {\it i.e.}\ the propagator, varies with the renormalization scale $M$ in such a way that the bare parameters of the Lagrangian is unchanged. For the example of QED, the Callan-Symanzik equation for a Green's function with $n$ electron fields and $m$ photon fields is
\begin{align}
	\left[ M \frac{\partial}{\partial M} + \beta(e) \frac{\partial}{\partial e} + n\gamma_2(e) + m\gamma_3(e)\right] G^{(n,m)}(\{ x_i\}; M,e) = 0.\label{eq:callansymanzik}
\end{align}
The functions $\beta$ and $\gamma$ are defined as
\begin{align}
	\beta \equiv \frac{M}{\delta M} \delta e, \, \gamma_i \equiv - \frac{M}{\delta M}\delta \eta_i,
\end{align}
where $\delta\eta_i$ is the field-strength renormalization term, shifting the field values of the electron and photon fields,
\begin{align}
	\psi \to (1 + \delta \eta_2) \psi \, \mathrm{and} \, A_\mu \to (1 + \delta\eta_3) A_\mu,
\end{align}
respectively. The Callan-Symanzik equation states that the combined effect of all the shifts in parameters induced by the renormalization should exactly weigh up for the shift in the Green's function itself, which is given by
\begin{align}
	G^{(n,m)} \to (1 + n\delta\eta_2 + m\delta\eta_3)G^{(n,m)}.
\end{align}
This is what is stated in eq.\ \eqref{eq:callansymanzik}. The Callan-Symanzik equation for other interactions, such as the $SU(3)$ quantum chromodynamics, may be derived similarily, but its complexity naturally grows with the complexity of the interaction.

The primary quantities of interest from a phenomenological viewpoint are the $\beta$ and $\gamma$ functions. They describe the change in the coupling constant and other parameters as a function of renormalization scale, and in the case of QED they may be used to derive the formula \eqref{eq:electron_charge_running} for the running of the electromagnetic coupling constant $e$. Equation \eqref{eq:electron_charge_running} shows that the electromagnetic coupling constant increases as a function of the energy $q$. The same turns out to be true for the weak coupling constant, while the strong coupling constant of QCD decreases with increasing energy. This last fact is called {\it asymptotic freedom}, and means that the quarks and gluons are unbound by strong forces in the limit of high energy. 





\section{Motivations for extending the Standard Model}

Since the Standard Model is widely believed to be a low-energy effective model of some more fundamental high-energy regime, it is speculated that the three interactions of the Standard Model unite at a higher energy and act as a single interaction under some larger gauge group. However, when the three couplings are evolved to high energies using the Callan-Symanzik equations, they do not meet at a single point. This is seen by many as a flaw of the Standard Model. In the theory of Supersymmetry, the evolution of the couplings is altered, and they do meet at a single point. This is shown in fig. \ref{fig:coupling_unification}. Supersymmetry is discussed in detail in the next chapter.
\begin{figure}[hbt]
	\centering
	\includegraphics[width=0.6\textwidth]{figures/susyintro/unification.eps}
	\caption{Evolution of the inverse coupling constants, for the cases of the Standard Model (dashed lines) and models with supersymmetry (solid lines). From \cite{Martin:1997ns}.}
	\label{fig:coupling_unification}
\end{figure}

Another issue with the Standard Model is that is has no candidate for particle dark matter. Observations over the last century have given strong evidence for the existence of some as yet unkown form of matter which is distributed in large quantites all over the universe -- in fact four times as much as our ordinary matter. It is widely believed that this dark matter is some form of particle. Dark matter interacts primarily, or possibly even solely, via gravitation, so the particle has to be colourless and electrically neutral, because the strength of these interactions would otherwise lead to the particle having been observed by now. It also has to be long-lived in order to explain the abundance of dark matter that we observe in the universe today, because the assumption is that it was thermally produced in the early universe and subsequently cooled off. These restrictions rule out most of the Standard Model particles, with the exception of neutrinos. But neutrinos are known to be very light, almost massless, and calculations of early-universe dynamics show that they are too light to be candidates for dark matter. 

There is also a more technical problem with the Standard Model, related to the scalar Higgs field. As discussed in section \ref{subsec:feynmancalculus}, the calculations of parameters in a quantum field theory are subject to loop corrections. The Higgs mass parameter recieves corrections from loops containing all massive particles, with the largest contribution coming from the top quark. These contributions are many orders of magnitude larger than the observed Higgs mass of 126 GeV, meaning that there must be cancellations among the correction terms. There is no symmetry in the Standard Model which says that such a cancellation should occur, so it appears to be an ``accident'' of nature. Such accidents are seen as very unnatural, and this explanation is thus very unsatisfactory from a theoretical viewpoint. This is called the {\it hierarchy problem}. In supersymmetry, the new degrees of freedom enter into the loop corrections, and because of the symmetry, they cancel the Standard Model contributions in a natural way. 



% The Standard Model is a quantum field theoretic model and may be stated in terms of a Lagrangian density function $\mathcal{L}$. The features that define the Standard Model emerge by requiring that it is invariant under the action of certain {\it gauge groups} -- specifically the infamous $U(1)_Y\times SU(2)_L\times SU(3)_C$, where the subscripts stand for {\it hypercharge, left} and {\it colour}, respectively, and refer to the properties that the different fields must have in order to be acted upon by the group transformations. By inserting certain {\it fermionic} field content in the Lagrangian and imposing these symmetries, the model acquires a number of {\it gauge bosons} for the different gauge groups. All these particles are {\it a priori} massless, a requirement to fulfill the gauge symmetry. To give particles their observed mass, then, one adds a scalar field and a corresponding scalar potential of a certain shape. The shape of the potential is such that the $U(1)_Y\times SU(2)_L$ symmetry is {\it spontaneously broken} at a certain energy scale, shifting the degrees of freedom around to give a scalar Higgs boson along with the other gauge bosons, and also giving mass to all particles except the photon. The remaining unbroken symmetries are then the $U(1)_\mathrm{em}$ for electromagnetic and $SU(3)_C$ for strong interactions. 



% The particles that make up the Standard Model are: the three generations of charged leptons, electron, muon and tau, and their three neutral counterparts, the neutrinos; the three generations of up- and down-type quarks up, down, charm, strange, top and bottom; the electroweak gauge bosons photon, Z and W; the strong gauge bosons, the gluons; and the Higgs boson. 





%%%%%%%%%%%%%%%%%%%%%%%%%%%%%%%%%%%%%%%%%%%%%%%%%%%%%%%%%%%%%%%%%%%%%%%%
\chapter{Supersymmetry}%%%%%%%%%%%%%%%%%
%%%%%%%%%%%%%%%%%%%%%%%%%%%%%%%%%%%%%%%%%%%%%%%%%%%%%%%%%%%%%%%%%%%%%%%%
\label{ch:susyintro}
The theory of supersymmetry (SUSY) is a proposed extension of the Standard Model which increases the number of degrees of freedom by introducing a symmetry between fermions and bosons, called a supersymmetry. The construction of supersymmetry is in some sense a two-step process, where one first derives the Lagrangian of a theory with complete symmetry between fermions and bosons, meaning that every bosonic degree of freedom gets a corresponding `supersymmetric' fermionic degree of freedom, and {\it vice versa}. These fields only differ in spin. But since {\it e.g.}\ scalar, colour charged particles with the same mass as the quarks are not observed, the symmetry cannot be exact. To make the theory physically viable, the supersymmetric partners must be significantly heavier than their Standard Model counterparts. This means that the supersymmetry must be a broken symmetry, and this breaking is put into the theory ``by hand''.

In this chapter we will outline the construction of a supersymmetric theory. First, we introduce the group theoretic framework of the symmetries. We define the concept of superfields, fields transforming under representations of the supersymmetry group. We go on to construct a fully supersymmetric Lagrangian in the framework of the Minimal Supersymmetric Standard Model (MSSM). Then the breaking of SUSY is achieved by manually inserting so-called ``soft'' SUSY-breaking terms. Also, the concept of R-parity is introduced in order to ensure the stability of the proton. R-parity will also make the lightest supersymmetric particle a good dark matter candidate. From the broken SUSY Lagrangian, we extract the particle content -- identifying the familiar fields of the Standard Model as well as their supersymmetric counterparts. We then introduce popular phenomenological models used to constrain and study the parameter space of the MSSM, and discuss their implications for the hierarchy of SUSY masses. This type of constrained model is subsequently adapted for the study of particular chain decays, which is the topic for the remainder of the thesis. We will also review the current experimental status of SUSY. The presentation lends inspiration from \cite{Batzing:2013} and \cite{Leinonen:2014}.

\section{Extending the Poincar\'{e} symmetry}
In Chapter \ref{ch:SM_intro}, the Poincar\'{e} group was discussed. It is the group of all Lorentz boosts and rotations, as well as all translations in spacetime. Any physical theory obeying Special Relativity must be invariant under the Poincar\'{e} group. It was shown in 1967 by Coleman and Mandula \cite{PhysRev.159.1251} that there exists no extension of the Poincar\'{e} symmetry which includes the gauge groups of the Standard Model in a non-trivial way, {\it i.e.}\ a way by which the extended group cannot be decoupled as a direct product such that the groups do not couple to each other. 

This prompted Haag, Łopuszański and Sohnius (HLS) \cite{Haag1975257} to introduce the concept of a {\it superalgebra}. A superalgebra is a direct product of two Lie algebras with a binary operation that mixes the algebras together. HLS constructed a superalgebra by combining the Lie algebra of the Poincar\'{e} group with an algebra spanned by four operators called {\it Majorana spinor charges}, represented by a two-component Weyl spinor $Q_A$ (to be defined shortly) and its hermitian conjugate $\bar Q_{\dot A}$. The resulting superalgebra is given by the (anti)commutation relations
\begin{align}
	[Q_A,P_\mu] =& [\bar Q_{\dot A}] = 0,\\
	[Q_A, M_{\mu\nu}] =& \sigma_{\mu\nu,A}^B Q_B,\\
	\{Q_A, Q_B\} =& \{\bar Q_{\dot A}, \bar Q_{\dot B} \} = 0,\\
	\{Q_A, \bar Q_{\dot B} \} =& 2\sigma^\mu_{A \dot B} P_\mu,\\
\end{align}
where $P_\mu$ is the momentum operator (the generator of translations in the Poincar\'{e} group), $M_{\mu\nu}$ are the generators of Lorentz boosts and rotations, $\sigma_\mu = (1,\sigma_i)$ with $\sigma_i$ the Pauli matrices and $\sigma_{\mu\nu} = \frac{i}{4}(\sigma_\mu \bar\sigma_\nu - \sigma_\nu \bar\sigma_\mu)$. 

In the usual representation of the Poincar\'{e} group, the fermion fields are represented as four-component Dirac spinors. It can be shown that the Poincar\'{e} group is isomorphic to $SL(2,\mathbb{C})$, so it is possible to work with representations of this group instead. The $SL(2,\mathbb{C})$ group has two inequivalent fundamental representations by two-component spinors which are called {\it left- and right-handed Weyl spinors} and written as $\psi_A$ and $\bar\psi_{\dot A}$, respectively. 

\section{Superfields}
The representations of the objects transforming under the supersymmetry transformations are called superfields. There are two important types, called chiral/scalar and vector superfields. The superfields are expressed in terms of spacetime coordinates $x^\mu$ and four anti-commuting Grassman numbers $\theta_A$ and $\bar\theta_{\dot A}$. Together, these eight coordinates form the {\it superspace}. Because of the anticommutativity, which means that any Grassman number squared vanishes, a function of a Grassman number, $f(\theta_A)$, has an all-order expansion given by
\begin{align}
	f(\theta) = a + b\theta_A.
\end{align}
Using this fact, a superfield $\Phi$ may generally be written as
\begin{align}
	\Phi = &f(x) + \theta^A\phi_A(x) + \bar\theta_{\dot A}\bar\chi^{\dot A}(x) + \theta \theta m(x)\\
	 &+ \bar\theta \bar\theta n(x) + \theta\sigma^\mu \bar\theta V_\mu(x) + \theta\theta\bar\theta_{\dot A}\bar\lambda^{\dot A}(x) + \bar\theta \bar\theta \theta^A \psi_A(x) + \theta \theta \bar\theta \bar\theta d(x).\nonumber
\end{align}
The different field components have the following properties: $f(x)$, $m(x)$ and $n(x)$ are complex (pseudo)scalars, $\psi_A(x)$ and $\phi_A(x)$ are left-handed Weyl spinors, $\bar\chi^{\dot A}(x)$ and $\bar\lambda^{\dot A}(x)$ are right-handed Weyl spinors, $V_\mu (x)$ is a Lorentz four-vector and $d(x)$ is a complex scalar.
% \begin{itemize}
% 	\item $f(x), m(x)$ and $n(x)$ are complex (pseudo)scalars
% 	\item $\psi_A(x)$ and $\phi_A(x)$ are left-handed Weyl spinors
% 	\item $\bar\chi^{\dot A}(x)$ and $\bar\lambda^{\dot A}(x)$ are right-handed Weyl spinors
% 	\item $V_\mu (x)$ is a Lorentz four-vector
% 	\item $d(x)$ is a complex scalar
% \end{itemize}

 A set of covariant derivatives are defined by
\begin{align}
	D_A = \frac{\partial}{\partial \theta^A} - i\sigma^\mu_{A \dot A}\bar\theta^{\dot A}\partial_\mu, \, \bar D^{\dot A} = -\frac{\partial}{\partial \bar \theta_{\dot A}} + i\bar\sigma^{\mu,A \dot A}\theta_A \partial_\mu.
\end{align}
In terms of these, a left chiral superfield $\Phi$ is defined by the condition.
\begin{align}
	\bar D^{\dot A} \Phi = 0.
\end{align}
By substituting $y^\mu = x^\mu - i\theta\sigma^\mu \bar \theta$, the covariant derivative $\bar D^{\dot A}$ is given as
\begin{align}
	\bar D^{\dot A} = -\frac{\partial}{\partial \bar\theta_{\dot A}}.
\end{align}
This shows that a left-chiral superfield must be independent of $\bar \theta$ in these coordinates, so it may generally be written as
\begin{align}
	\Phi(y, \theta) = A(y) + \sqrt{2}\theta\psi(y) + \theta\theta F(y), 
\end{align}
thus containing two complex scalar fields and a left-handed Weyl spinor. Under an infinitesimal supersymmetry transformation
\begin{align}
	\delta_\xi \Phi = -i(\xi Q + \bar\xi\bar Q)\Phi,
\end{align}
the component field $F$ can be shown to transform into a total derivative. It will thus not contribute to the action, since all fields must vanish on the boundary at infinity. For this reason it is called an auxillary field. Thus we see that a left-chiral superfield contains two bosonic (scalar) degrees of freedom and two fermionic degrees of freedom contained in a left-handed Weyl spinor. Similar arguments may be applied to define a right-chiral superfield by the condition
\begin{align}
	D_A \Psi^\dag = 0,
\end{align}
and to show that it contains two auxillary and two proper scalar degrees of freedom, as well as a right-handed Weyl spinor. 

A {\it vector superfield} is defined by the condition
\begin{align}
	\Phi^\dag = \Phi.
\end{align}
This condition allows the field content, for a general vector superfield $V$,
\begin{align}
	V = &f(x) + \theta^A\phi_A(x) + \bar\theta_{\dot A}\bar\chi^{\dot A}(x) + \theta \theta m(x) + \bar\theta \bar\theta m^*(x)\\
	 &+ \theta\sigma^\mu \bar\theta V_\mu(x) + \theta\theta\bar\theta_{\dot A}\bar\lambda^{\dot A}(x) + \bar\theta \bar\theta \theta^A \lambda_A(x) + \theta \theta \bar\theta \bar\theta d(x).
\end{align}
Here, the scalar fields $f(x)$ and $d(x)$, as well as the four-vector $V_\mu (x)$, are required to be real fields, thus halving their amount of degrees of freedom. There are auxillary degrees of freedom which may be removed by a gauge transformation. A vector superfield may be written in the {\it Wess-Zumino} gauge as
\begin{align}
	V_\mathrm{WZ} = (\theta \sigma^\mu \bar\theta) \left[ V_\mu(x) + i\partial_\mu (A(x) - A^*(x)) \right] + \theta\theta \bar\theta_{\dot A} \bar\lambda^{\dot A}(x) + \bar\theta \bar\theta \theta_A \lambda^A(x) + \theta\theta \bar\theta \bar\theta d(x).
\end{align}
In this gauge, the vector superfield contains one real scalar field degree of freedom (d.o.f.), three gauge field d.o.f.'s and four fermion d.o.f.'s.

\section{The unbroken SUSY Lagrangian}
To obtain a theory which is supersymmetric, the action, given by
\begin{align}
 	S = \int d^4 x \mathcal{L},
 \end{align}
 needs to be invariant under SUSY transformations. As mentioned in the previous section, a total derivative has this property because it is determined by the boundary conditions, where it has to vanish. It can be shown that the highest-order component fields in $\theta$ and $\bar \theta$, {\it i.e.}\ the term proportional to $\theta\theta\bar\theta\bar\theta$, always has this property for both chiral and vector superfields and products thereof. Thus the invariance of the action may be ensured by redefining the Lagrangian such that
 \begin{align}
 	S = \int d^4 x \int d^4 \theta \mathcal{L},
 \end{align}
 where the last integral is over the four Grassman variables. This will project out only the desired terms, because of how the Grassman integral is defined. Thus the supersymmetric Lagrangian may be constructed from superfields and their products.

 A {\it superpotential} is defined as a product of left-chiral superfields,
 \begin{align}
 	W(\Phi) = L^i\Phi_i + \frac{1}{2}m^{ij}\Phi_i\Phi_j + \frac{1}{3}\lambda^{ijk}\Phi_i\Phi_j\Phi_k.
 \end{align}
The inclusion of higher-order field terms are forbidden from the condition of renormalizability, which forbids terms where the combined mass dimension of the fields are larger than four, since scalar, fermionic and auxillary fields have mass dimension one, 3/2 and 2, respectively. A fourth order superfield term would include field terms which break this condidtion. The most general Lagrangian that can be written in terms of chiral superfields is
\begin{align}
	\mathcal{L} = \Phi_i^\dag \Phi_i + \bar\theta\bar\theta W(\Phi) + \theta\theta W(\Phi^\dag),
\end{align}
where the first term is called the kinetic term. 

The Lagrangian has to be gauge invariant. The general gauge transformation of a chiral superfield is given by
\begin{align}
	\Phi \overset{G}{\to} e^{-iq\Lambda^a T_a}\Phi
\end{align}
where $T_a$ are the group generators and $q$ is the charge of $\Phi$ under $G$ and $\Lambda_a$ themselves can be shown to be left-chiral superfields. The equivalent transformation for a right-chiral superfield $\Phi^\dag$ involves a right-chiral superfield gauge parameter $\Lambda_a^\dag$.

Analogously to the Standard Model, the supersymmetric gauge interactions are introduced as compensating terms to the gauge transformation of the chiral superfields. The analogue to the gauge boson fields are the vector superfields $V^a$, which are introduced into the kinetic terms of the Lagrangian by writing them as
\begin{align}
	\Phi^\dag_i e^{q V^a T_a}\Phi_i,
\end{align}
such that the kinetic term transforms as
\begin{align}
	\Phi_i^\dag e^{q V^a T_a}\Phi_i \overset{G}{\to} \Phi^\dag e^{iq(\Lambda^a)^\dag T_a}e^{qV^{'a}T_a}e^{-iq\Lambda^a T_a}\Phi,
\end{align}
which is invariant given that the vector superfields transform as
\begin{align}
	e^{qV^{'a} T_a} = e^{-iq(\Lambda^a)^\dag T_a}e^{qV^a T_a}e^{iq\Lambda^a T_a}.
\end{align}
For infinitesimal $\Lambda$, this is to leading order
\begin{align}
	V^{'a} = V^a 0 i(\Lambda^a - (\Lambda^a)^\dag) - \frac{1}{2}qf^a_{bc} V^b ((\Lambda^c)^\dag + \Lambda^c).
\end{align}
This gives for the vector component fields of the vector superfields, $V_\mu^a$,
\begin{align}
	V^a_\mu \overset{G}{\to} V^{'a}_\mu = V_\mu^a + i\partial_\mu(\Lambda^a - (\Lambda^a)^\dag) - qf^a_{bc} V_\mu^b (\Lambda^c + (\Lambda^{c})^\dag).
\end{align}
With these definitions, it can be shown that the Standard Model couplings of fermions with bosons are recovered by defining the covariant derivative
\begin{align}
	D_\mu^i = \partial_\mu - \frac{i}{2} q_i V_\mu.
\end{align}

The SUSY Lagrangian terms containing the field strengths of the gauge fields are written as
\begin{align}
	\mathrm{Tr}[W^A W_A],
\end{align}
called the supersymmetric field strength, where $W_A$ and $\bar W_{\dot A}$ are left- and right-handed chiral superfields, respectively, given by
\begin{align}
	W_A &\equiv -\frac{1}{4}\bar D\bar D e^{-qV^aT_a} D_A e^{qV^a T_a},
	\bar W_{\dot A} &\equiv -\frac{1}{4} D D e^{-qV^aT_a} \bar D_{\dot A} e^{qV^a T_a}.
\end{align}
The general form of the SUSY Lagrangian is
\begin{align}
	\mathcal{L} \Phi^\dag e^{qV^a T_a}\Phi + \bar\theta\bar\theta W(\Phi) + \theta\theta W(\Phi^\dag) + \frac{1}{2T(R)}\bar\theta \mathrm{Tr}(W^A W_A),
\end{align}
where $T(R)$, the {\it Dynkin index} of the representation of the gauge group, is a normalization constant.



\subsection{SUSY breaking}
Supersymmetry has to be a broken theory, at least in the low-energy limit, since supersymmetric particles with Standard Model masses are not observed. The breaking can be inserted into the SUSY Lagrangian ``by hand'', by explicitly adding terms that break SUSY and allow for mass splitting. The rationale for these terms is that the SUSY Lagrangian is only an effective Lagrangian where some heavy field has been integrated out, and that the breaking of SUSY occurs through this field at a higher scale. There are several alternatives for the mechanisms of SUSY breaking, some of which are Planck-scale mediated SUSY breaking, gauge mediated SUSY breaking and anomaly mediated SUSY breaking. Whichever of the mechanisms is chosen, there are only a finite set of terms that may be added to the Lagrangian without breaking renormalizability. They are called {\it soft} SUSY breaking terms, required to have couplings of mass dimension one or higher, and may in the most general form be written
\begin{align}
	\mathcal{L}_\mathrm{soft} = &-\frac{1}{4T(R)}M\theta\theta\bar\theta\bar\theta \mathrm{Tr} [W^A W_A] - \frac{1}{6}a_{ijk} \theta\theta\bar\theta\bar\theta\Phi_i \Phi_j \Phi_k\nonumber\\
	&-\frac{1}{2}b_{ij} \theta\theta\bar\theta\bar\theta\Phi_i \Phi_j - t_i \theta\theta\bar\theta\bar\theta \Phi_i + \mathrm{h.c.}\\
	&-m_{ij} \theta\theta\bar\theta\bar\theta \Phi_i^\dag \Phi_j.\nonumber
\end{align}
In terms of the component fields of the superfields, the soft Lagrangian may be written
\begin{align}
	\mathcal{L}_\mathrm{soft} = &-\frac{1}{2} M\lambda^A\lambda_A - \left( \frac{1}{6} a_{ijk} A_i A_j A_k + \frac{1}{2} b_{ij} A_i A_j + t_i A_i + \frac{1}{2} c_{ijk} A^*_i A_j A_k + \mathrm{c.c.}\right)\\
	&- m_{ij}^2 A_i^* A_j.
\end{align}
Since this contains both Weyl spinor fields and scalar fields, the soft terms may be used to modify masses and couplings of the supersymmetric scalar and fermionic fields which will appear in a moment.



\section{The Minimal Supersymmetric Standard Model}
The Minimal Supersymmetric Standard Model (MSSM) is the minimal supersymmetric theory which contains the Standard Model. It is constructed by choosing field content in accordance with the requirements of the previous sections. To construct a Dirac fermion, we use one left-chiral and one right-chiral superfield together. This gives the four fermionic degrees of freedom that a Dirac fermion and its antiparticle requires. Since each chiral superfield also contains two scalar degrees of freedom (after removing the auxillary fields), this introduces two scalar particle-antiparticle pairs, which are called the supersymmetric partners, or {\it superpartners}, of the Dirac fermion. An important point is that all superfield components must have the same charge under all gauge groups, due to the way the gauge transformation was defined. This means that the scalar fields generally will be charged. For instance, the superfields for the charged leptons are denoted $l_i$ and $\bar E_i$ for the left- and right-chiral superfields, respectively, and the left-handed neutrino superfield is denoted $\nu_i$. Here, $i=1,2,3$ is a generation index. The $SU(2)_L$ doublet of the Standard Model is recovered by setting $L_i = (\nu_i, l_i)$. The quark superfields are denoted $u_i$, $\bar U_i$, $d_i$ and $\bar D_i$, where $Q_i = (u_i, d_i)$ makes the $SU(2)_L$ doublet.

The gauge boson fields come from the vector superfields, each of which also contains two Weyl-spinor fields of opposite handedness. To obey gauge invariance, $n^2-1$ vector superfields are required for each of the $SU(n)$ groups just as in the Standard Model -- {\it i.e.}\ gauge invariance under $U(1)_Y\times SU(2)_L \times SU(3)_C$ requires 1+3+8 vector superfields. These are denoted $B^0$, $W^a$ and $C^a$, respectively. The Weyl spinor fields, corresponding to superpartners of the gauge fields, are written as $\tilde B^0$, $\tilde W^0$ and $\tilde g$, respectively. In the literature, these are often referred to as {\it bino}, {\it wino} and {\it gluino}. 

The MSSM requires two Higgs superfield $SU(2)_L$ doublets to be able to give mass to both up- and down-type quarks. In the Standard Model, the same Higgs doublet can be used for both types by rotating the components using the $SU(2)_L$ generators, but this is not possible in SUSY. The Higgs doublets in the MSSM are
\begin{align}
	H_u = \begin{pmatrix}
		H_u^+ \\ H_u^0
	\end{pmatrix}, \, H_d = \begin{pmatrix}
		H_d^0 \\ H_d^-
	\end{pmatrix}.
\end{align}
This introduces several additional Higgs scalars into the model, as well as the fermionic superpartner fields.

The fields listed above come together and construct the MSSM Lagrangian $\mathcal{L}_\mathrm{MSSM}$ using the rules described above for a general gauge invariant SUSY Lagrangian, giving rise to kinetic terms, superpotential terms and supersymmetric field strength terms. The Lagrangian will {\it a priori} contain terms which break lepton and baryon number conservation, such as $LH_u, LLE, LQ\barD \in \mathcal{L}_\mathrm{MSSM}$. If the couplings for these terms is large, it allows for proton decay, which is experimentally very heavily constrained, with a lifetime of $\tau_\mathrm{proton} > 10^{33} \,\mathrm{yr}$. To avoid this, it is conventional to introduce the concept of {\it R-parity}.

\subsection{R-parity}
R-parity is a multiplicative quantum number which is assumed to be conserved in all SUSY interactions. Formally, a particle has R-parity given by
\begin{align}
	R = (-1)^{2s + 3B + L},
\end{align}
where $s$ is the particle's spin, $B$ its baryon number and $L$ its lepton number. The important point is that all Standard Model particles have $R=+1$ while all superpartners have $R = -1$. This leads to the very important prediction that superpartner particles only can be produced and annihilated in pairs. In particular, it means that the lightest superpartner (LSP) must be stable against decay. This makes the LSP very attractive as a candidate for dark matter, if it is electrically neutral.

\subsection{Radiative electroweak symmetry breaking}
Like any quantum field theory, the MSSM is subject to renormalization, which induces the running of the coupling constants and masses of the model as discussed in Chapter \ref{ch:SM_intro}. In particular, the mass parameters $m_{H_u/d}$ for the Higgs doublets, which come from soft breaking terms in the Lagrangian, run with energy. To break the electroweak symmetry, it is assumed that these are equal at some high scale, and run down. It can be shown that the running is proportional to the Yukawa couplings of the third-generation quarks of the respective type -- {\it i.e.}\ the top and bottom, respectively. Because the top Yukawa coupling is much larger than the bottom one, the $m_{H_u}$ parameter runs down much faster with energy and becomes negative. One can further show that this is a sufficient condition to induce the electroweak symmetry breaking, and thus give the gauge bosons and fermions their observed mass.

\subsection{Particle phenomenology of the MSSM}
The total particle content of the MSSM is as follows: 
\begin{itemize}
	\item The Standard Model particles are present: electrons, muons, taus and their corresponding neutrinos; the up, down, strange, charm, bottom and top quarks; the photon, Z boson and W bosons and gluons; and the Higgs boson.
	\item In addition to the Standard Model Higgs $h$, there are four other scalar Higgs particles with positive R-parity, labeled $H$, $H^\pm$ and $A$. $H$ is identical to $h$ except for its larger mass, and is therefore termed ``heavy Higgs'', in contrast to the ``light Higgs'' of the Standard Model. The other neutral field $A$ is a pseudo-scalar.
	\item All the Standard Model particles get superpartners:
	\begin{itemize}
	 	\item For the gluons, they are called gluinos and labeled $\tilde g$. 
	 	\item The partners of the $B^0$ and $W^a$ fields, which in the Standard Model form the photon, $Z$ and $W^\pm$, mix with the superpartner Higgs fields to form four neutral Majorana fermions called neutralinos, labeled $\chi_i^0$, $i=1,...,4$, and two charged particle-antiparticle pairs called charginos, $\chi_i^\pm$, $i=1,2$. 
	 	\item Each of the Standard Model fermions get two corresponding scalar particles with the same gauge charges. For the first two generations, the mass eigenstates split to good approximation into one left-chiral and one right-chiral fermion, such that {\it e.g.}\ the superpartners of the up quark $u$ are labeled $\tilde u_R$ and $\tilde u_L$. For the third-generation fermions, the chiral approximation is cruder, so these mass eigenstates are just numbered, {\it e.g.}\ $\tilde b_1$ and $\tilde b_2$ for the $b$ quark. 
	 \end{itemize} 
 \end{itemize}

\subsection{The Constrained MSSM}

\begin{itemize}
	\item Include plot from Stephen Martin. 
\end{itemize}

\begin{figure}[hbt]
	\centering
	\includegraphics[width=0.8\textwidth]{figures/susyintro/MSSMrun.eps}
	\caption{MSSM RGE running, from \cite{Martin:1997ns}.}
	\label{fig:mssm_rgerun}
\end{figure}



























%%%%%%%%%%%%%%%%%%%%%%%%%%%%%%%%%%%%%%%%%%%%%%%%%%%%%%%%%%%%%%%%%%%%%%%%%%%
\chapter{Determination of SUSY particle masses from cascade decays}%%%%%%%%
%%%%%%%%%%%%%%%%%%%%%%%%%%%%%%%%%%%%%%%%%%%%%%%%%%%%%%%%%%%%%%%%%%%%%%%%%%%
\label{ch:introducing_the_method}


\marginpar{Separate into introduction chapter}
In the following chapters we will present and discuss a method for determining the masses of supersymmetric particles in certain types of cascade decays. The present chapter formulates the type of process we are studying and the problems we face, and presents a novel method proposed by B. Webber \cite{Webber:2009vm} for making inferences about the unknown masses. The subsequent chapters deal with investigation and improvements on the method. We begin by simulating events, adding complexities layer by layer, investigating the method and getting some feeling for its aptitude. We then discuss some problems with the original formulation, suggest ways to amend these and look at ways to develop the method further.

\section{The problem}
Consider an LHC event where two chains of the form
\begin{align}
	\tilde{q} \to q + \tilde{\chi}_2^0, \, \tilde{\chi}_2^0 \to l^{\pm} + \tilde{l}^\mp, \, \tilde{l}^\mp \to l^\mp + \tilde{\chi}_1^0\label{eq:goldencascade}
\end{align}
are present. Combined, the measurable particles in the two chains are the two quarks and four leptons, where the lepton pairs are opposite-sign same-flavour. The LSPs escape detection, but the sum of their transverse momenta can be measured as the missing $p_T$. The quantities of interest, however, are the masses of the supersymmetric particles, $m_{\tilde{q}}, m_{\tilde{\chi}_2^0}, m_{\tilde{l}}$ and $m_{\tilde{\chi}_1^0}$. (Potentially with several values for the squarks and sleptons if they differ in generation between the sides.) These are not directly measurable, but the kinematics of the problem depend on them. 

Many methods have been investigated for the purpose of measuring supersymmetric masses \cite{Barr:2010zj}. One well known example is the end-point method \cite{1126-6708-2000-09-004}. We measure {\it e.g.}\ the dilepton invariant mass in the process \eqref{eq:goldencascade}. The distribution of the dilepton invariant mass can be shown to form a right triangle where the maximal value is given by
\begin{align}
	(m_{ll}^\mathrm{max})^2 = \frac{ \left( m^2_{\tilde{\chi}_2^0} - m^2_{\tilde{l}} \right) \left( m^2_{\tilde{l}} - m^2_{\tilde{\chi}_1^0} \right)}{m^2_{\tilde{l}}}, \label{eq:invariant_mass_endpoint}
\end{align}
thus constraining $m_{\tilde{\chi}_2^0}$, $m_{\tilde{l}}$ and $m_{\tilde{\chi}_1^0}$. Similar constraints may be obtained for the three other visible particle combinations, giving four equations with four unknowns. This method is very dependent on statistics, since each measured event only contributes one point to the distribution. A large number of events is required to get a reliable value. However, the number of events contributing to the measurement is also much larger, since each side of the decay chain contribute individually, thus making use of events with leptons on one side only as well.



\section{Webber's method}
Webber \cite{Webber:2009vm} suggests a different method, where all available kinematical info from every event is used. Consider the general decay tree in fig. \ref{fig:decaytree}. We assume that we have an event with two such chains, but not necessarily with identical particles in the two. We will distinguish the two chains by referring to them as the <<first>> and <<second>> one, although the assignment is arbitrary.
\begin{figure}[hbt]
\centering
\includegraphics[scale=0.7]{figures/fig-chain.pdf} % Stolen!
\caption{Decay topology.}
\label{fig:decaytree}
\end{figure}
Assuming that the decaying particles are on-shell, the four-momenta in the first chain should satisfy
\begin{align}
	(p_c + p_b + p_a + p_A)^2 &= M_D^2,\nonumber \\
	(p_b + p_a + p_A)^2 &= M_C^2,\nonumber \\
	(p_a + p_A)^2 &= M_B^2,\label{eq:constraints}\\
	p_A^2 &= M_A^2.\nonumber
\end{align}
The first three equations give three linear constraints on the invisible four-momentum $p_A$:
\begin{align}
	-2p_c\cdot p_A &= M_C^2 - M_D^2 + 2p_c\cdot p_b + 2p_c \cdot p_a + m_c^2 \equiv S_1,\nonumber \\
	-2p_b\cdot p_A &= M_B^2 - M_C^2 + 2p_b\cdot p_a + m_b^2 \equiv S_2,\\
	-2p_a\cdot p_A &= M_A^2 - M_B^2 + m_a^2 \equiv S_3. \nonumber
\end{align}
Equivalently the second chain (with primed indices) gets the constraints
\begin{align}
	-2p_{c'}\cdot p_{A'} &= M_{C'}^2 - M_{D'}^2 + 2p_{c'}\cdot p_{b'} + 2p_{c'} \cdot p_{a'} + m_{c'}^2 \equiv S_5,\nonumber \\ 
	-2p_{b'}\cdot p_{A'} &= M_{B'}^2 - M_{C'}^2 + 2p_{b'}\cdot p_{a'} + m_{b'}^2 \equiv S_6,\\
	-2p_{a'}\cdot p_{A'} &= M_{A'}^2 - M_{B'}^2 + m_{a'}^2 \equiv S_7.\nonumber
\end{align}
In addition we have the transverse momentum constraints
\begin{align}
	p_A^x + p_{A'}^x &= p_\mathrm{miss}^x \equiv S_4, \label{eq:Svec_orig} \\
	p_A^y + p_{A'}^y &= p_\mathrm{miss}^y \equiv S_8. \nonumber
\end{align}

The vector $\mathbf{S} = (S_1, S_2, ...)$ thus depends on the eight unknown masses 
\begin{align}
	\mathbf{M} = (M_D^2, M_C^2, M_B^2, M_{A'}^2, M_{D'}^2, M_{C'}^2, M_{B'}^2, M_{A'}^2)
\end{align}
and the visible momenta which in principle are measurable. We define a vector containing the four-momenta of the invisible final state particles as
\begin{align}
	\mathbf{P} = (p_A^x, p_A^y, p_A^z, E_A, p_{A'}^x, p_{A'}^y, p_{A'}^z, E_{A'}). \label{eq:Pvec}
\end{align}
We then have
\begin{align}
	\mathbf{A}\mathbf{P} = \mathbf{S},\label{eq:APS}
\end{align}
where
\begin{align}
	\mathbf{A} = 2 \begin{pmatrix}
						p_c^x & p_c^y & p_c^z & -E_c & 0 & 0 & 0 & 0 \\
						p_b^x & p_b^y & p_b^z & -E_b & 0 & 0 & 0 & 0 \\
						p_a^x & p_a^y & p_a^z & -E_a & 0 & 0 & 0 & 0 \\
						1/2 & 0 & 0 & 0 & 1/2 & 0 & 0 & 0\\
						0 & 0 & 0 & 0 & p_{c'}^x & p_{c'}^y & p_{c'}^z & -E_{c'} \\
						0 & 0 & 0 & 0 & p_{b'}^x & p_{b'}^y & p_{b'}^z & -E_{b'} \\
						0 & 0 & 0 & 0 & p_{a'}^x & p_{a'}^y & p_{a'}^z & -E_{a'} \\
						0 & 1/2 & 0 & 0 & 0 & 1/2 & 0 & 0
					\end{pmatrix}. \label{eq:Amatrix_orig}
\end{align}
Furthermore, $\mathbf{S}$ may be written as 
\begin{align}
	\mathbf{S} = \mathbf{B} \mathbf{M} + \mathbf{C},\label{eq:SBMC}
\end{align}
where
\begin{align}
	\mathbf{B} = \begin{pmatrix}
					-1 & 1 & 0 & 0 & 0 & 0 & 0 & 0 \\
					0 & -1 & 1 & 0 & 0 & 0 & 0 & 0 \\
					0 & 0 & -1 & 1 & 0 & 0 & 0 & 0 \\
					0 & 0 & 0 & 0 & 0 & 0 & 0 & 0 \\
					0 & 0 & 0 & 0 & -1 & 1 & 0 & 0 \\
					0 & 0 & 0 & 0 & 0 & -1 & 1 & 0 \\
					0 & 0 & 0 & 0 & 0 & 0 & -1 & 1 \\
					0 & 0 & 0 & 0 & 0 & 0 & 0 & 0 \\
	\end{pmatrix}
\end{align}
and
\begin{align}
	\mathbf{C} = ( &2p_c \cdot p_b + 2p_c \cdot p_a + m_c^2, 2 p_2 \cdot p_3 + m_b^2, m_a^2, p_\mathrm{miss}^x, \nonumber \\ 
				   &2p_{c'}\cdot p_{b'} + 2 p_{c'} \cdot p_{a'} + m_{c'}^2, 2 p_{b'} \cdot p_{a'} + m_{b'}^2, m_{a'}^2, p_\mathrm{miss}^y )
\end{align}
With all this, the solution for the invisible four-momenta, given the unknown masses, is 
\begin{align}
	\mathbf{P} = \mathbf{A}^{-1} \mathbf{S} = \mathbf{D} \mathbf{M} + \mathbf{E},
\end{align}
where $\mathbf{D} = \mathbf{A}^{-1}\mathbf{B}$ and $\mathbf{E} = \mathbf{A}^{-1}\mathbf{C}$.

The matrix $\mathbf{D}$ and vector $\mathbf{E}$ contain only measurable quantities, hence they only need to be calculated once for every event. For the true value of the unknown masses $\mathbf{M}$, the system should satisfy the on-shell conditions
\begin{align}
	p_{A}^2 &= P_4^2 - P_1 ^2 - P_2^2 - P_3^2 = M_{A}^2, \nonumber\\
	p_{A'}^2 &= P_8^2 - P_5 ^2 - P_6^2 - P_7^2 = M_{A'}^2.
\end{align}
So by calculating $\mathbf{D}_n$ and $\mathbf{E}_n$ for each event $n$, and making a {\it hypothesis} $\mathbf{M}$ for the unknown masses, we can measure the goodness of fit for our hypothesis by the quantity
\begin{align}
	\xi^2(\mathbf{M}) = \sum_n \left[(p_{A}^2)_n - M_A^2\right]^2 + \left[(p_{A'}^2)_n - M_{A'}^2\right]^2. \label{eq:xisquared}
\end{align}
Note that this quantity measures the goodness-of-fit of all the unknown masses equally, since it follows from the constraint equations \eqref{eq:constraints} that {\it e.g.}
\begin{align}
	(p_B^2)_n - M_B^2 &= (p_a + p_A)_n^2 - M_B^2 = \nonumber\\
				  &= (p_a^2)_n + (p_A^2)_n + 2p_a\cdot p_A - M_B^2\nonumber\\
				  &= (p_a^2)_n + (p_A^2)_n - M_A^2 + M_B^2 - m_a^2 - M_B^2\\
				  &= (p_A^2)_n - M_A^2.\nonumber
\end{align}

For each event there are eight constraining equations. There are eight unknowns from the masses plus six from the unknown LSP momenta (using the on-shell constraint on the LSPs). The system is thus underconstrained, with six free parameters. The point of the method is to minimize $\xi^2$ as a function of $\mathbf{M}$. This is generally an eight-dimensional minimization problem with a potentially very complicated function, and thus not easy to solve. However, in the case of identical chains, it reduces to a much more handleable four-dimensional one which one could hope could be solved. In this case the number of free parameters reduces from six to two. The condition of identical chains can often be satisfied by a combination of vetoing ({\it e.g.} b-jets) and assuming small mass splittings between different generations, thus approximating their masses as equal. This is a realistic assumption in many SUSY scenarios.\marginpar{Refer to previous chapter about SUSY}

\section{Two technical modifications}\label{sec:dimension_fixing}
The aptness of the method hangs on the invertibility of the matrix $\mathbf{A}$.\marginpar{Does it really hang on it?} However, the matrix as it stands in \eqref{eq:Amatrix_orig}, is technically ill-defined for inversion since not all rows have the same units. The rows 4 and 8, corresponding to the components 4 and 8 of the vector $\mathbf{S}$ \eqref{eq:Svec_orig}, have no dimension, while the other rows have dimension $(\mathrm{mass})^1$. This is reflected in the components of $\mathbf{S}$, which all except 4 and 8 have dimension $(\mathrm{mass})^2$. This means both that the magnitude of the determinant is sensitive to the choice of mass scale (since some rows have non-zero dimension) and that it does not scale properly for numerical calculations (since not all rows have the same dimension). This is something that Webber does not comment on (and the method still in principle works), but we make some minor reformulations of the method in order to amend both problems.

For the first, we redefine $S_4$ and $S_8$ to be 
\begin{align}
	S_4 &\equiv (p_A^x + p_{A'}^x)^2 = (p_\mathrm{miss}^x)^2, \label{eq:Svec_modified} \\
	S_8 &\equiv (p_A^y + p_{A'}^y)^2 = (p_\mathrm{miss}^y)^2. \nonumber
\end{align}
We do not wish to redefine $\mathbf{P}$ \eqref{eq:Pvec}, so to keep the relationship $\mathbf{S} = \mathbf{A}\mathbf{P}$ we modify rows 4 and 8 of $\mathbf{A}$ to
\begin{align}
	\mathbf{A}_4 &= (p_\mathrm{miss}^x, 0, 0, 0, p_\mathrm{miss}^x, 0, 0, 0),\\
	\mathbf{A}_8 &= (0, p_\mathrm{miss}^y, 0, 0, 0, p_\mathrm{miss}^y, 0, 0),\nonumber
\end{align}
such that $\mathbf{A}$ now is
\begin{align}
	\mathbf{A} = 2 \begin{pmatrix}
						p_c^x & p_c^y & p_c^z & -E_c & 0 & 0 & 0 & 0 \\
						p_b^x & p_b^y & p_b^z & -E_b & 0 & 0 & 0 & 0 \\
						p_a^x & p_a^y & p_a^z & -E_a & 0 & 0 & 0 & 0 \\
						p_\mathrm{miss}^x/2 & 0 & 0 & 0 & p_\mathrm{miss}^x/2 & 0 & 0 & 0\\
						0 & 0 & 0 & 0 & p_{c'}^x & p_{c'}^y & p_{c'}^z & -E_{c'} \\
						0 & 0 & 0 & 0 & p_{b'}^x & p_{b'}^y & p_{b'}^z & -E_{b'} \\
						0 & 0 & 0 & 0 & p_{a'}^x & p_{a'}^y & p_{a'}^z & -E_{a'} \\
						0 & p_\mathrm{miss}^y/2 & 0 & 0 & 0 & p_\mathrm{miss}^y/2 & 0 & 0
					\end{pmatrix}. \label{eq:Amatrix_modified}
\end{align}

This redefinition does not alter the solvability of the problem, since the only information lost in $\mathbf{S}$ is the sign of $p_\mathrm{miss}^i$ which is kept in $\mathbf{A}$ instead. Also it keeps the essential feature that $\mathbf{A}$ only contains measured quantities, so that it can be inverted prior to making a mass hypothesis. The redefinition of $\mathbf{S}$ means we also have to modify $\mathbf{C}$ to keep the relationship $\mathbf{S} = \mathbf{B} \mathbf{M} + \mathbf{C}$ (from eq.\ \eqref{eq:SBMC}). We thus make the same redefinitions here, {\it i.e.}
\begin{align}
	C_4 &\equiv (p_\mathrm{miss}^x)^2, \label{eq:Cvec_modified} \\
	C_8 &\equiv (p_\mathrm{miss}^y)^2. \nonumber
\end{align}

The other issue is to make the numerical problem dimensionless. All elements of $\mathbf{A}$ and $\mathbf{P}$ now have mass dimension 1, while all elements of $\mathbf{S}$, and thus $\mathbf{M}$ and $\mathbf{C}$, have dimension 2. We are free to multiply both sides of eq.\ \eqref{eq:APS} by some normalization mass $M_\mathrm{norm}$ squared,
\begin{align}
	\frac{1}{M_\mathrm{norm}^2} \mathbf{A}\mathbf{P} = \frac{1}{M_\mathrm{norm}^2} \mathbf{S},
\end{align}
and we choose to take it into the matrix and vectors such that they all become dimensionless, {\it i.e.}\ we modify
\begin{align}
	\mathbf{\hat A} = \frac{1}{M_\mathrm{norm}}\mathbf{A},\nonumber \\
	\mathbf{\hat P} = \frac{1}{M_\mathrm{norm}}\mathbf{P},\label{eq:vectors_normalized}\\
	\mathbf{\hat S} = \frac{1}{M_\mathrm{norm}^2}\mathbf{S},\nonumber 
\end{align}
thus modifying $\mathbf{M}$ and $\mathbf{C}$ in the same way as $\mathbf{S}$ to comply with eq.\ \eqref{eq:SBMC}. We also modify the fitting function $\xi^2$ accordingly, so that it becomes
\begin{align}
	\xi^2(\mathbf{M}) = \sum_n \left[(\hat p_{A}^2)_n - \frac{M_A^2}{M_\mathrm{norm}^2}\right]^2 + \left[(\hat p_{A'}^2)_n - \frac{M_{A'}^2}{M_\mathrm{norm}^2}\right]^2.\label{eq:xisquared_modified}
\end{align}

To obtain numbers of order one, which is optimal for numerical purposes, we should pick a mass of the relevant scale for the problem. This is not something that is known {\it a priori}, since it depends on the supersymmetric masses that we are trying to determine. We might be tempted to use something based on the measured momenta, but this is a bad idea since it would mean weighting different events differently. We choose the normalization constant
\begin{align}
	M_\mathrm{norm} = 100 \,\mathrm{GeV},
\end{align}
the same order of magnitude as we expect for the supersymmetric masses ($\sim$ electroweak scale). 

We have made thorough checks that this formulation and the original one produce identical results within numerical accuracy, so that indeed the formulations are equivalent.









\section{Taking account of combinatorics}
\label{sec:combinatorics}
In a real detector event, the ordering of the quarks and leptons in and between chains is not known -- all we have are the measured particle types and their momenta. We must take this into account when applying the method to Monte Carlo simulated datasets. Webber does this by evaluating all possible combinations in each event at each mass point and selecting the combination which gives the lowest $\xi^2$ value, choosing to add this value to the sum in eq.\ \eqref{eq:xisquared_modified}. The number of possible combinations are 8 or 16, depending on whether the lepton pairs in the two chains are the same flavour or not. 

For two pairs of different-generation leptons, the possible orderings are (given some `base ordering' which we permute from): Switching the ordering of the leptons in the first chain; switching the ordering of the leptons in the second chain; or switching the leptons in both chains. For each of these permutations we have the option to switch the two quarks, so the total number of combinations is 8. In the case of identical leptons, we may additionally interchange leptons between the chains -- but this only increases the total combinations by a factor of 2 because the same-chain leptons must have opposite charge.

Note that in order to switch the ordering of leptons within the same chain, all we need to do is permute the rows of the matrix $\mathbf{A}$. The vector $\mathbf{C}$ is invariant as long as the same-chain leptons have the same mass. When the quarks are flipped, however, or when leptons are interchanged between chains, then we must redefine $\mathbf{A}$ and $\mathbf{C}$. Webber makes a point that this property can save computer memory, since one only has to store two or four versions of the matrix and vector for each event. We have not found this to be an issue.












\section{Outline of the plan}
\marginpar{Better reformulate away from ``reproduce'', and also modify to be what I actually do}
In this thesis we wish to investigate and develop this method further. We will begin by trying to reproduce Webber's parton level results using Monte Carlo simulations. We will then add layers of realism and complexety approaching something closer to the real experimental situation, in order to investigate its full potential. In the end we will focus on well motivated scenarios that can be discovered in Run II of the LHC.  Along the way we will also make some improvements on the method.

The plan of attack is as follows:
\begin{enumerate}
	\item We begin by generating squark pairs at rest, decaying them in the chain of on-shell two-body decays given in (\ref{eq:goldencascade}). The visible decay products, quarks and leptons, are then used to reconstruct the masses. As a benchmark we investigate the precision attainable for the parameter point SPS1a~\cite{Allanach:2002nj}, which was used by Webber. 
	\item Because of final state radiation and parton showering, as well as---to a lesser degree---sparticle widths, the particles in the decay chain will not be on-shell due to extra gluons (and photons) in the final state. We employ a more sophisticated Monte Carlo code, {\tt Herwig++}~\cite{Bahr:2008pv}  to simulate these properties. This should have an effect on the mass reconstruction since Webber's method assumes on-shell decays.
	\item	We then compare results with and without including the combinatorical issues from identifying the decay products, and we add a simple parametrised momentum smearing, based on realistic detector response, in order to simulate that the measurement of the kinematics of final-state particles is not exact. This was the level of precision employed by Webber in~\cite{Webber:2009vm}.
	\item The partons that emerge from the hadron showers will hadronize, forming a hadron jet before arriving in the detector. Measurement of the initial parton from reconstructing such jets is one of the big challenges of collider physics. We use the {\tt FastJet}~\cite{Cacciari:2011ma} program for jet reconstruction, with algorithms used in the LHC experiments, and study the effect of jet reconstruction on the mass measurement.
	\item In an analysis of real data, one would have to use selection criteria such as cuts on transverse momentum and number of jets and leptons to discriminate between signal and background events. We will apply such cut schemes to our simulated events based on expectations for 14 TeV LHC. In addition we simulate the expected largest backgrounds for a four-lepton supersymmetry search and investigate how the performance of the method is affected by the presence of background.
	\item {\bf If time} As a last step towards realism, we put the events through a fast detector simulation. 
	\item We then investigate improvements over the original model. Webber and collaborators \cite{Nojiri:2010dk} propose combining the kinematical best-fit reconstruction with measurements of end points of invariant mass distributions.
	\item Investigating other types of chains. Not-equal-sided? Different number of steps? Fitting a mass plane rather than points?
	\item Finally, we look at a well motivated model taken from \cite{Allanach:2014gsa} that was constructed on the basis of a small excess seen in 8 TeV data by the CMS Collaboration~\cite{CMS:2014jfa}, which would be consistent with the decay chain in (\ref{eq:goldencascade}). We show what precission we can expect from mass measurments at 14 TeV LHC should such a scenario be realized.
\end{enumerate}






























%%%%%%%%%%%%%%%%%%%%%%%%%%%%%%%%%%%%%%%%%%%%%%%%%
\chapter{Investigating Webber's method by Monte Carlo simulations}
\label{ch:MC}
%%%%%%%%%%%%%%%%%%%%%%%%%%%%%%%%%%%%%%%%%%%%%%%%%
Webber demonstrates the aptitude of the method on a Monte Carlo generated dataset. A natural starting point for our study is to try to reproduce his results. 

\section{Collider physics}
There are several ways in which a pair of squarks can be produced in $pp$ collisions. The three main categories are: direct pair production of two squarks or a squark-antisquark pair; squark plus gluino with subsequent gluino decay to a squark; and pair production of two gluinos which both subsequently decay to squarks. The mechanism of production affects how much momentum goes into the squark, and thus the momentum of the subsequent decay chain. But our method deals mainly with the internal kinematics of the squark decay, which by Lorentz invariance is independent of squark momentum. The only variables in our analysis which explicitly depends on the overall event kinematics is the missing transverse momentum. The three different categories also determine how many hard jets are present in the event, which affects the combinatorical aspects of reconstructing the chain. This will be discussed in detail later.\marginpar{Better remember to discuss it in detail.}

\section{Reproducing Webber's results}
Webber uses Fortran {\tt HERWIG} version 6.510 \cite{Corcella:2000bw,Moretti:2002eu} to produce events, selecting only events with two left-handed first- or second generation squarks (to limit the mass splitting), but irrespective of the hard production process. The analysis, {\it i.e.} minimization of the $\xi^2$, is performed on 100 samples of 25 events each, using the Minuit {\tt Simplex} \cite{James:1975dr} routine for minimization. He models the effects of measurement errors in a real detector by applying momentum smearing according to a gaussian distribution, and he puts a cut on the total $\xi^2$ obtained at the minimum to eliminate samples which give a bad result. His results are summarized in table 1 of \cite{Webber:2009vm}, which we for convenience show in table \ref{table:webber_original}. 
\begin{figure}[hbt]
	\centering
	\includegraphics[width=0.6\textwidth]{figures/webber_rec_table/sps1a_fits.eps} 
	\caption{fig.\ 2 from \cite{Webber:2009vm}, displayed here for comparison.}
	\label{fig:webber_scatter}
\end{figure}
The column $\delta p/p$ indicates the standard deviation of the gaussian smearing applied on the momenta, $\xi^2_\mathrm{max}$ indicates the cut value of the $\xi^2$, $f_\xi$ is the fraction of samples surviving the cut and $f_\mathrm{cor}$ is the fraction of events where the chosen particle combination, selected as described in section \ref{sec:combinatorics}, is the correct one. 


\begin{table}[hbt]
	\centering
	\begin{tabular}{| l | l | l | l  || l | l | l | l |}
		\hline
		$\delta p/p$ & $\xi^2_\mathrm{max}$ & $f_\xi$ & $f_\mathrm{cor}$ & $m_{\tilde q} (540)$ & $m_{\tilde \chi_2^0} (177)$ & $m_{\tilde l} (143)$ & $m_{\tilde \chi_1^0} (96)$ \\
		\hline \hline
		0 & 	$\infty$ &	100 \%	& 72 \%	& $538 \pm 20$	&	$176 \pm 12$	&	$143 \pm 7$	& 	$95 \pm 10$	\\
		0 &		100 &		80 \%	& 76 \% & $539 \pm 7$	&	$177 \pm 1$		&	$144 \pm 1$	&	$96 \pm 2$	\\
		5 \% &	$\infty$ &	100 \%	& 52 \% & $534 \pm 28$	& 	$176 \pm 11$	&	$143 \pm 10$&	$95 \pm 13$ \\
		5 \% &	100 &		57 \%	& 55 \% & $539 \pm 9$	&	$178 \pm 3$		& 	$144 \pm 2$	&	$96 \pm 4$	\\
		10 \% &	$\infty$ &	100 \%	& 40 \% & $522 \pm 37$	&	$171 \pm 18$	&	$140 \pm 17$&	$88 \pm 26$	\\
		10 \% &	200 &		42 \%	& 43 \% & $530 \pm 22$	& 	$173 \pm 12$	&	$140 \pm 12$&	$89 \pm 20$ \\
		\hline
	\end{tabular}
	\caption{Webber's table of results, taken from table 1 of \cite{Webber:2009vm}.}
	\label{table:webber_original}
\end{table}



% We have made a thorough investigation of these results using a variety of tools. We have generated Monte Carlo events using both Herwig++ 2.7.1  \cite{Bahr:2008pv} and Pythia 8.2 \cite{Sjostrand:2014zea}.\marginpar{Make use of Pythia or don't mention?} We have used the Simplex algorithm in various versions as well as other minimization algorithms for the fit. Webber himself also very generously sent us his own MC generation code, so we were able to make a quite close reconstruction of his analysis. But in the process of confirming his results, we have discovered what appears to be an error in the analysis. The error has to do with the choice of tolerance in the minimization routine. To understand this, we have to briefly discuss the Simplex algorithm.

To reproduce Webber's results, we have generated events using {\tt Herwig++ 2.7.1} \cite{Bahr:2008pv}. As a control, and to enable us to interface the simulation with detector simulation software later, we have also used {\tt Pythia 8.2} \cite{Sjostrand:2014zea}.\marginpar{Remove if I don't do detector simulation.} We have also had access to the code used in the original paper, enabling us to make a quite close reconstruction of Webber's analysis \cite{Webber:epost}. 

To minimize the $\xi^2$ function, we have used the {\tt Simplex} algorithm. We have applied the Minuit version as well as a custom implementation listed in appendix \ref{ch:simplex}. We have discovered that the mass fit is heavily dependent on the input parameters to the {\tt Simplex} minimization, and this makes the fit more challenging. To facilitate the subsequent discussion, we briefly introduce the Simplex method.

\section{The Nelder-Mead Simplex algorithm}

{\tt Simplex} \cite{nelder1965simplex} is a heuristic minimization search method for minimizing a scalar function of $N$ variables. It takes a starting parameter point as input from the user. From this parameter point it erects a {\it simplex}, an ``$N+1$-dimensional triangle''. It then begins to evaluate the function in the vertices of this simplex. A new simplex is constructed by reflecting the vertex with the highest function value around the (hyper)line made out of the other $N$ vertices. Hopefully this new simplex lies lower in the function terrain, and thus the algorithm iterates towards a local minimum. In case of trouble, it may also try contracting the simplex or distorting its shape in various ways to obtain points of lower function values. 

Since the method is heuristic, so is the convergence criterion. Convergence is said to be obtained when the {\it estimated distance to minimum (EDM)} is smaller than some set tolerance value. Usually there is also a predefined maximal number of iterations before the method gives up, to avoid going on forever on non-converging problems. The EDM is defined as
\begin{align}
	\mathrm{EDM}(f_\mathrm{min},f_\mathrm{max}) = \frac{|f_\mathrm{max}-f_\mathrm{min}|}{|f_\mathrm{max}| + |f_\mathrm{min}|},
\end{align}
where $f_\mathrm{min}$ and $f_\mathrm{max}$ are the function values at the lowest and highest point of the current simplex, respectively. This means that the convergence criterion really measures how ``flat'' the simplex is, and thus how steep the function is in the region. If the tolerance is too high, then, we run the risk of obtaining convergence in a region where the gradient is not steep enough to be resolved by the set tolerance, but which may still be far from the minimum.

A pitfall of any minimization routine, also for Minuit Simplex, is that it has a default tolerance value which is used automatically unless the user specifically changes it. The default tolerance in Minuit Simplex is 0.1. This appears to be what Webber has used. We have confirmed that we obtain statistically consistent results when choosing that value. But this does not always resolve this particular function well enough, because it tends to have almost flat directions in mass space for some sets of events.\marginpar{Include $\xi^2$ surface plots. Discuss a bit more?} It therefore leads to convergence at a non-minimal point. If, additionally, the search is started at or close to the masses used to generate the Monte Carlo, then the minimization may obtain ``convergence'' at points very close to the true value, but these points are not minimal points, just regions where the function is not very steep.

\section{The tolerance is too high}

Refer to the scatter plot in fig.\ 2 of \cite{Webber:2009vm}, displayed in fig.\ \ref{fig:webber_scatter} for convenience. This scatter plot shows the fit corresponding to the first row of table \ref{table:webber_original}. We have reproduced this fit using Webber's code \cite{Webber:epost} -- albeit with one modification: Since we don't have access to the old {\tt ISAJET} and {\tt ISAWIG} software for RGE running of SUSY parameters, we have generated our SPS 1a parameter point using {\tt SoftSUSY} version 3.4.1 \cite{Allanach:2001kg}, and converted the resulting SLHA \cite{Skands:2003cj} model file to {\tt ISAWIG} format using the package {\tt PySLHA} version 3.0.2\footnote{We have had to make several modifications to the {\tt PySLHA} code to make the {\tt ISAWIG} output readable by {\tt HERWIG}. These changes have been included in {\tt PySLHA} 3.0.3.} \cite{Buckley:2013jua}. The effects of using a different RGE runner is that the SUSY masses are not exactly the same. The most significant shift is the squarks, which in Webber's case have a mass of 540 GeV, compared to 565 GeV in our case. We observe similar results as Webber gives in his article when we run his code with the original settings for our SUSY parameter point. 
\begin{figure}[hbt]
	\centering
	\begin{subfigure}[b]{0.6\textwidth}
		\includegraphics[width=\textwidth]{figures/webber_rec_table/webber_rec_table-samesettings_0psmear-nocut.pdf} 
		\caption{ }
		\label{fig:webber_rec_scatter_tolerance-comparison_a}
	\end{subfigure}

	\begin{subfigure}[b]{0.6\textwidth}
		\includegraphics[width=\textwidth]{figures/webber_rec_table/webber_HW-rec_nocut.pdf}
		\caption{ } 
		\label{fig:webber_rec_scatter_tolerance-comparison_b}
	\end{subfigure}
	\caption{Reproduction of Webber's results (corresponding to fig. \ref{fig:webber_scatter} and the first row of table \ref{table:webber_original}) for (a) original convergence tolerance and (b) a lower tolerance criterion.}
	\label{fig:webber_rec_scatter_tolerance-comparison}
\end{figure}
Our reproduction of fig.\ \ref{fig:webber_scatter} is shown in fig.\ \ref{fig:webber_rec_scatter_tolerance-comparison_a}, and our reproduction of table \ref{table:webber_original} is given in table \ref{table:webber_softsusy}.
\begin{table}[hbt]
	\centering
	\begin{tabular}{| l | l | l | l  || l | l | l | l |}
		\hline
		$\delta p/p$ & $\xi^2_\mathrm{max}$ & $f_\xi$ & $f_\mathrm{cor}$ & $m_{\tilde q} (540)$ & $m_{\tilde \chi_2^0} (177)$ & $m_{\tilde l} (143)$ & $m_{\tilde \chi_1^0} (96)$ \\
		\hline \hline
		0 & 	$\infty$ &	100 \%	& 65 \%	& $566 \pm 9$	&	$180 \pm 2$		&	$144 \pm 2$	& 	$97 \pm 3$	\\
		0 &		100 &		85 \%	& 67 \% & $567 \pm 6$	&	$180 \pm 1$		&	$144 \pm 1$	&	$97 \pm 3$	\\
		5 \% &	$\infty$ &	100 \%	& 43 \% & $564 \pm 26$	& 	$181 \pm 14$	&	$145 \pm 10$&	$94 \pm 15$ \\
		5 \% &	100 &		52 \%	& 48 \% & $566 \pm 10$	&	$180 \pm 2$		& 	$145 \pm 2$	&	$96 \pm 4$	\\
		10 \% &	$\infty$ &	100 \%	& 33 \% & $551 \pm 33$	&	$180 \pm 15$	&	$144 \pm 11$&	$91 \pm 24$	\\
		10 \% &	200 &		43 \%	& 36 \% & $559 \pm 17$	& 	$177 \pm 11$	&	$143 \pm 11$&	$91 \pm 20$ \\
		\hline
	\end{tabular}
	\caption{Our reproduction of table \ref{table:webber_original}, using Webber's code \cite{Webber:epost} with original settings, except with the masses from SoftSUSY.}
	\label{table:webber_softsusy}
\end{table}







However, the tolerance setting in {Minuit} can be adjusted. When we rerun the code used to produce fig.\ \ref{fig:webber_rec_scatter_tolerance-comparison_a} with the tolerance set to $10^{-12}$, we get the fit shown in \ref{fig:webber_rec_scatter_tolerance-comparison_b}. The results are not dramatically altered, but there are some features to notice: There is a clear tendency to a linear correlation between the masses. This is a feature we should expect physically: If one reduces one of the masses, {\it e.g.}\ the squark, then this should affect the fit of the other masses, reducing them correspondingly.\footnote{This is part of the reason why these kinds of mass reconstruction methods very often reconstruct the squared mass {\it difference} rather than the masses themselves.} The fact that this physically reasonable degenerate direction appears when the tolerance is reduced indicates that a such a reduction is necessary to achieve reliable results. We also note that the fitted masses now seem slightly biased toward lower values. Finally we note that while the mean value and errors are still consistent with the true values, their accuracy is somewhat reduced. Particularly so for the LSP, where the fit is reduced from $99 \pm 3$ GeV to $83 \pm 19$ GeV, compared to the true value of 97 GeV.

These fit results, with the low tolerance setting, are still not bad. However, in table \ref{table:webber_original}, Webber also gives best-fit values where he has applied smearing to the four-momenta, as a crude approximation to the effects of limited detector resolution. The momentum smearing is done by smearing the spatial components according to a gaussian distribution of a given r.m.s. width $\delta p/p$, and then defining the energy in such a way that the invariant mass is unchanged \cite{Webber:epost}.
\begin{figure}[hbt]
	\centering
	\begin{subfigure}[b]{0.6\textwidth}
		\includegraphics[width=\textwidth]{figures/webber_rec_table/webber_HW-rec_OFL_minuit-minimizer_hightol_5pmomsmear_nocut.pdf} 
		\caption{ }
	\end{subfigure}

	\begin{subfigure}[b]{0.6\textwidth}
		\includegraphics[width=\textwidth]{figures/webber_rec_table/webber_HW-rec_OFL_minuit-minimizer_lowtol_5pmomsmear_nocut.pdf}
		\caption{ } 
	\end{subfigure}
	\caption{Reproduction of Webber's 5\% momentum-smeared fit (corresponding to the third row of table \ref{table:webber_original}) for (a) original convergence tolerance and (b) a lower tolerance criterion.}
	\label{fig:webber_rec_scatter_tolerance-comparison_5pmomsmear}
\end{figure} 
In fig.\ \ref{fig:webber_rec_scatter_tolerance-comparison_5pmomsmear} we show scatter plots of the fits to the dataset smeared with $\delta p/p = 5 \%$, minimized with original and reduced tolerance, again using Webber's code for event generation and minimization. The fit with original tolerance is consistent with fig.\ 3 of \cite{Webber:2009vm}, as it should be. However, when the tolerance is reduced, the fit results are worsened considerably. Since each event is smeared individually, this appears to greatly affect the position of the minimum. Again we see that the LSP (yellow) recieves the roughest treatment, being pushed to much lower values than the true one in most cases. The results are even worse for the 10 \% smeared dataset. We also note that in the HERWIG code Webber uses, initial-state radiation of photons in SUSY decays is switched off, which may contribute to making the measurement unrealistically good.\marginpar{Are doesn't think this is a big effect, but I am not sure. I did complete momentum conservation in my events same as Webber, in practice turning off ISR. Should consider checking, or rewriting.}




% \begin{figure}[hbt]
% 	\centering
% 	\begin{subfigure}[b]{0.6\textwidth}
% 		\includegraphics[width=\textwidth]{figures/webber_rec_table/webber_HW-rec_OFL_minuit-minimizer_hightol_10pmomsmear_nocut.pdf} 
% 		\caption{ }
% 	\end{subfigure}

% 	\begin{subfigure}[b]{0.6\textwidth}
% 		\includegraphics[width=\textwidth]{figures/webber_rec_table/webber_HW-rec_OFL_minuit-minimizer_lowtol_10pmomsmear_nocut.pdf}
% 		\caption{ } 
% 	\end{subfigure}
% 	\caption{Reproduction of Webber's 10 \% momentum-smeared fit (corresponding to fig. 3 of \cite{Webber:2009vm} and the fifth row of table \ref{table:webber_original}) for (a) original convergence tolerance and (b) a lower tolerance criterion.}
% 	\label{fig:webber_rec_scatter_tolerance-comparison_10pmomsmear}
% \end{figure}

However, Webber also investigates the effects of imposing a {\it cut} on the $\xi^2$ value obtained at the minimum. In his fits, this cut tends to remove the bad events, giving a better fit. Applying a cut also helps for the reduced-tolerance fit, although it does not recover Webber's original low error estimate. In table \ref{table:webber_rec_lowtol} we reproduce table \ref{table:webber_original} for the reduced-tolerance fit. We note that the fraction of samples passing the $\xi^2$ cut is drastically reduced compared to table \ref{table:webber_original}. The fraction of events where the best-fit combination is the true one is also reduced by about half. 

\begin{table}[hbt]
	\centering
	\begin{tabular}{| l | l | l | l  || l | l | l | l |}
		\hline
		$\delta p/p$ & $\xi^2_\mathrm{max}$ & $f_\xi$ & $f_\mathrm{cor}$ & $m_{\tilde q} (568)$ & $m_{\tilde \chi_2^0} (180)$ & $m_{\tilde l} (144)$ & $m_{\tilde \chi_1^0} (97)$ \\
		\hline \hline
		0 & 	$\infty$ &	100 \%	& 36 \%	& $563 \pm 13$	&	$173 \pm 10$	&	$136 \pm 11$	& 	$83 \pm 19$	\\
		0 &		100 &		35 \%	& 52 \% & $565 \pm 9$	&	$175 \pm 8$		&	$138 \pm 9$	&	$86 \pm 16$	\\
		5 \% &	$\infty$ &	100 \%	& 31 \% & $557 \pm 27$	& 	$165 \pm 17$	&	$125 \pm 15$&	$58 \pm 27$ \\
		5 \% &	100 &		13 \%	& 43 \% & $558 \pm 14$	&	$164 \pm 11$	& 	$126 \pm 12$	&	$65 \pm 22$	\\
		10 \% &	$\infty$ &	100 \%	& 29 \% & $542 \pm 35$	&	$158 \pm 20$	&	$116 \pm 17$&	$36 \pm 28$	\\
		10 \% &	200 &		15 \%	& 33 \% & $549 \pm 20$	& 	$155 \pm 12$	&	$116 \pm 12$&	$38 \pm 25$ \\
		\hline
	\end{tabular}
	\caption{Reproduction of the fits in table \ref{table:webber_softsusy}, but with reduced convergence tolerance.}
	\label{table:webber_rec_lowtol}
\end{table}

\section{The starting-point dependence of the best-fit point}
\label{sec:SP-dependence_webber}

There is also another potential issue with Webber's analysis. It has to do with the fact that the best-fit search is started at the true mass values.  In a real experiment, these parameters are exactly the unknowns we wish to find. Starting the search here is in principle fine as long as we are sure that the algorithm always finds the true global minimum. So we must investigate what happens if we start our search in some other point. We have done this, and discover that this greatly affects the location of the best-fit points.
\begin{figure}[hbt]
	\centering
	\begin{subfigure}[b]{0.45\textwidth}
		\includegraphics[width=\textwidth]{figures/webber_rec_table/webber_HW-rec_nocut.pdf} 
		\caption{ }
	\end{subfigure}
	\begin{subfigure}[b]{0.45\textwidth}
		\includegraphics[width=\textwidth]{figures/webber_rec_table/webber-rec_wrong_starting_point-400-300-200-100_lowtol.pdf}
		\caption{ } 
	\end{subfigure}

	\begin{subfigure}[b]{0.45\textwidth}
		\includegraphics[width=\textwidth]{figures/webber_rec_table/webber-rec_wrong_starting_point-800-500-300-50_lowtol.pdf} 
		\caption{ }
	\end{subfigure}
	\begin{subfigure}[b]{0.45\textwidth}
		\includegraphics[width=\textwidth]{figures/webber_rec_table/webber-rec_wrong_starting_point-1000-100-80-30_lowtol.pdf}
		\caption{ } 
	\end{subfigure}
	\caption{Minimization on the unsmeared {\tt HERWIG} dataset for different starting points: $\vec M = (568, 180, 144, 97)$ (the TMP) in {\bf (a)}, $\vec M = (400, 300, 200, 100)$  in {\bf (b)}, $\vec M = (800, 500, 300, 50)$ in {\bf (c)} and $\vec M = (1000, 100, 80, 30)$ in {\bf (d)}.}
	\label{fig:starting_point_sensitivity_combinatorics}
\end{figure}
In fig.\ \ref{fig:starting_point_sensitivity_combinatorics} we show the best-fit points for four low-tolerance minimizations on the {\tt HERWIG} dataset. Subfig.\ (a) is the same as \ref{fig:webber_rec_scatter_tolerance-comparison} (b), the minimization started from the true mass point (TMP). The other three are minimizations from other starting points, selected to illustrate other plausible mass spectra: one where both the masses and the mass differences are smaller (b); one where they are larger (c); and one where there is a large splitting between a heavy squark and three very light masses (d). It is obvious that the fit -- the location of the best-fit point, and also the number of samples where convergence is obtained, indicated by the number $N_\mathrm{bins}$ in each plot -- is hugely dependent on where we start the search. For instance, the mean value and standard errors of the squark masses for the samples range from $506 \pm 98 \mathrm{GeV}$ to $563 \pm 13 \mathrm{GeV}$. We also note that in the latter case, which is the minimization from the TMP, the mean values from the other three starting points fall outside the margin of error.\marginpar{Did I manage to get this sentence to make sense?}



It might, however, be that the function has multiple local minima, giving rise to the behaviour in fig.\ \ref{fig:starting_point_sensitivity_combinatorics}, but that the {\it global} minimum is the one we find by starting in the TMP. To investigate this, we have run minimizations (with low tolerance) where we perturb the starting point of each of the four parameters for each event sample away from the TMP by a gaussian distribution of width 10 or 20 GeV. This is a minor perturbation relative to the masses. The minimization results are shown in table \ref{table:webber_rec_lowtol_perturbedSP} for perturbations of 10 (top) and 20 (bottom) GeV. The standard errors of the fit increase considerably compared with \ref{table:webber_rec_lowtol}. \marginpar{Smear with 10 \% and 20 \% of the masses too?}
 
% we have made a scan of many starting points, minimizing the whole 100 bin dataset from each starting point. We selected five different values for each of the four SUSY masses and checked all combinations which make physical sense ({\it i.e.\ }where the mass hierarchy is present). This gave a total of 132 different starting points, which were indexed from 1 to 132. The TMP was point number 34. For each bin we then checked which index had the lowest $\xi^2$ value. Figure \ref{fig:webber_rec_hist_starting_points} shows a histogram of this. We see that number 34, the TMP, indeed dominates, but it is only the lowest in 14 of the 100 bins. The other 86 bins are distributed quite evenly among the other starting points. This leads us to conclude that the minimization, using this technique, simply is not well defined: There is no reasonable hope that we will be able to find a reliable best-fit estimate in an experimental situation where the TMP is unknown. This is especially true given that this scan was done on the unsmeared dataset -- we should expect even worse results when the resolution gets blurred by uncertainties.

% \begin{figure}[hbt]
% 	\centering
% 	\includegraphics[width=0.8\textwidth]{figures/webber_rec_table/histogram_of_starting_points_with_lowest_minimal_value.pdf}
% 	\caption{A histogram of 100 data bins minimized from different starting points, showing distribution of lowest $\xi^2$ values. See the text for details.}
% 	\label{fig:webber_rec_hist_starting_points}
% \end{figure}


\begin{table}[hbt]
	\centering
	\begin{tabular}{| l | l | l | l  || l | l | l | l |}
		\hline
		$\delta p/p$ & $\xi^2_\mathrm{max}$ & $f_\xi$ & $f_\mathrm{cor}$ & $m_{\tilde q} (568)$ & $m_{\tilde \chi_2^0} (180)$ & $m_{\tilde l} (144)$ & $m_{\tilde \chi_1^0} (97)$ \\
		\hline \hline
		10 GeV: & & & & & & & \\ 
		\hline
		0 & 	$\infty$ &	100 \%	& 47 \%	& $559 \pm 29$	&	$170 \pm 19$	&	$131 \pm 20$	& 	$68 \pm 38$	\\
		0 &		100 &		76 \%	& 48 \% & $564 \pm 12$	&	$175 \pm 13$		&	$137 \pm 14$	&	$83 \pm 25$	\\
		5 \% &	$\infty$ &	100 \%	& 38 \% & $557 \pm 24$	& 	$163 \pm 18$	&	$123 \pm 20$&	$52 \pm 36$ \\
		5 \% &	100 &		55 \%	& 37 \% & $559 \pm 16$	&	$168 \pm 13$	& 	$130 \pm 15$	&	$72 \pm 22$	\\
		10 \% &	$\infty$ &	100 \%	& 28 \% & $542 \pm 42$	&	$153 \pm 19$	&	$113 \pm 21$&	$19 \pm 31$	\\
		10 \% &	200 &		44 \%	& 26 \% & $550 \pm 21$	& 	$159 \pm 18$	&	$119 \pm 20$&	$38 \pm 36$ \\
		\hline
		20 GeV: & & & & & & & \\ 
		\hline
		0 & 	$\infty$ &	100 \%	& 42 \%	& $555 \pm 32$	&	$167 \pm 23$	&	$126 \pm 24$	& 	$60 \pm 41$	\\
		0 &		100 &		67 \%	& 44 \% & $564 \pm 14$	&	$173 \pm 14$	&	$135 \pm 16$	&	$82 \pm 24$	\\
		5 \% &	$\infty$ &	100 \%	& 34 \% & $550 \pm 45$	& 	$162 \pm 23$	&	$123 \pm 22$&	$46 \pm 39$ \\
		5 \% &	100 &		54 \%	& 34 \% & $554 \pm 23$	&	$163 \pm 21$	& 	$125 \pm 21$	&	$63 \pm 31$	\\
		10 \% &	$\infty$ &	100 \%	& 28 \% & $543 \pm 37$	&	$153 \pm 25$	&	$125 \pm 21$&	$16 \pm 32$	\\
		10 \% &	200 &		37 \%	& 27 \% & $539 \pm 22$	& 	$151 \pm 20$	&	$113 \pm 18$&	$24 \pm 31$ \\
		\hline
	\end{tabular}
	\caption{Reproduction of the fits in table \ref{table:webber_rec_lowtol} with perturbations of 10 and 20 GeV, respectively, on the starting points.}
	\label{table:webber_rec_lowtol_perturbedSP}
\end{table}



\clearpage

\section{Without combinatorics}

If we for a moment forget about the combinatorics, and evaluate only the true particle combination for each event, then the minimization gives consistent results irrespective of starting point. 
\begin{figure}[hbt]
	\centering
	\begin{subfigure}[b]{0.45\textwidth}
		\includegraphics[width=\textwidth]{figures/improving_combinatorics/herwigpp-momcons_nocomb_truemasspoint.pdf} 
		\caption{ }
	\end{subfigure}
	\begin{subfigure}[b]{0.45\textwidth}
		\includegraphics[width=\textwidth]{figures/improving_combinatorics/herwigpp-momcons_nocomb_400-300-200-100.pdf}
		\caption{ } 
	\end{subfigure}

	\begin{subfigure}[b]{0.45\textwidth}
		\includegraphics[width=\textwidth]{figures/improving_combinatorics/herwigpp-momcons_nocomb_800-500-300-50.pdf} 
		\caption{ }
	\end{subfigure}
	\begin{subfigure}[b]{0.45\textwidth}
		\includegraphics[width=\textwidth]{figures/improving_combinatorics/herwigpp-momcons_nocomb_1000-100-80-30.pdf}
		\caption{ } 
	\end{subfigure}
	\caption{An equivalent fit to fig. \ref{fig:starting_point_sensitivity_combinatorics} (on a {\ttfamily Herwig++} dataset), but the $\xi^2$ contribution is only evaluated for the true particle combination in each event.}
	\label{fig:starting_point_sensitivity_no_combinatorics}
\end{figure}
This is illustrated in fig.\ \ref{fig:starting_point_sensitivity_no_combinatorics}, which shows minimization of 100 samples of 25 events minimized with low tolerance, but only evaluated with the correct combination of chain particles.\footnote{This dataset is generated with {\tt Herwig++} version 2.7.1 \cite{Bahr:2008pv} and minimized using our own implementation of the {\tt Simplex} algorithm in C++, included in appendix \ref{ch:simplex}. Only events where the momentum in the chain is exactly conserved -- {\it i.e.}\ no final-state radiation of photons is allowed -- are included. We have checked that this dataset gives identical results with the Fortran {\tt HERWIG} program used in chapter \ref{ch:MC}.} There are differences between the fits, mainly between subfig.\ d and the others. In particular, while about 85 of the samples converge within the set limit of 500 iterations in each of the cases a, b and c (indicated by the value of $N_\mathrm{bins(total)}$ in the top right of each plot), this number is reduced to 67 in subfig.\ d. The starting point in case (d) is characterized by a much larger mass gap between the squark and the $\tilde\chi_2^0$ than in the TMP, which we might surmise would give rise to very different kinematics than we have in our events. In any case, the minimization is not a heavy calculation (the whole minimization of 100 bins, including combinatorics, is done in fractions of a second on a single CPU core), so doing a scan like the one we made above over different starting points is not unrealistic.\marginpar{Is there any point in doing a minimization grid scan here, for the no-combinatorics case? Maybe better to wait until I have handled combinatorics.}

We also keep in mind the effects of momentum smearing, and check the no-combinatorics minimization on the 5 \% smeared dataset for the same four starting points. The plots are shown in fig.\ \ref{fig:moving_on-starting_point_sensitivity_no_combinatorics_5pmomsmear}. We find that it also in this case gives consistent results irrespective of starting point -- but the LSP mass is fitted to zero in most samples.\marginpar{Count number of zero-fits} It appears that when the data is smeared, the method loses its sensitivity to the LSP mass. Or equivalently, we could say that it loses its sensitivity to the absolute mass scale of the problem. This is a well-known problem with methods for mass reconstruction.\marginpar{Might want to cite something if this sentence is kept, or elaborate about mass-squared difference fitting.}
\begin{figure}[hbt]
	\centering
	\begin{subfigure}[b]{0.45\textwidth}
		\includegraphics[width=\textwidth]{figures/improving_combinatorics/herwigpp_5psmear_lowtol_nocomb_TMP.pdf} 
		\caption{ }
	\end{subfigure}
	\begin{subfigure}[b]{0.45\textwidth}
		\includegraphics[width=\textwidth]{figures/improving_combinatorics/herwigpp_5psmear_lowtol_nocomb_400-300-200-100.pdf}
		\caption{ } 
	\end{subfigure}

	\begin{subfigure}[b]{0.45\textwidth}
		\includegraphics[width=\textwidth]{figures/improving_combinatorics/herwigpp_5psmear_lowtol_nocomb_800-500-300-50.pdf} 
		\caption{ }
	\end{subfigure}
	\begin{subfigure}[b]{0.45\textwidth}
		\includegraphics[width=\textwidth]{figures/improving_combinatorics/herwigpp_5psmear_lowtol_nocomb_1000-100-80-30.pdf}
		\caption{ } 
	\end{subfigure}
	\caption{Again the same fit as in \ref{fig:starting_point_sensitivity_combinatorics} and \ref{fig:starting_point_sensitivity_no_combinatorics}, here with a 5 \% smeared dataset and no combinatorics.}
	\label{fig:moving_on-starting_point_sensitivity_no_combinatorics_5pmomsmear}
\end{figure} 






























%%%%%%%%%%%%%%%%%%%%%%%%%%%%%%%%%%%%%%%%%%%%%%%
\chapter{Investigating improvements}%%%%%%%%%%%
%%%%%%%%%%%%%%%%%%%%%%%%%%%%%%%%%%%%%%%%%%%%%%%
With the potentially significant errors inherent in Webber's original suggestion, we will now turn to investigate improvements.


\section{Fitting mass squared differences}
We saw in the previous chapter that, even without taking combinatorical ambiguities into account, the method is insensitive to the absolute mass scale of the decay in many of the samples when the momentum resolution is smeared. In a later article \cite{Nojiri:2010dk}, Webber {\it et al} reformulate the method in terms of squared mass differences. We can borrow their idea and reformulate the problem as a mass-squared-difference fit. Such a fit may be combined with measurements of the dilepton invariant mass edge to find the LSP mass, using eq.\ \eqref{eq:invariant_mass_endpoint}, which can be rewritten as
\begin{align}
	m^2_{\tilde\chi_1^0} = (m^2_{\tilde l} - m^2_{\tilde \chi_1^0})\left(\frac{m^2_{\tilde\chi_2^0} - m^2_{\tilde l}}{(m_{ll}^\mathrm{max})^2} - 1\right),
\end{align}
or in the more abstract notation of fig.\ \ref{fig:decaytree},
\begin{align}
	M^2_A = (M^2_B - M^2_A)\left(\frac{M^2_C - M^2_B}{(m_{ll}^\mathrm{max})^2} - 1\right).\label{eq:MLSP_dilepton_edge}
\end{align}
Thus we see that the LSP mass can be found from knowing only the mass-squared differences plus the invariant mass edge. This idea is taken from \cite{Cheng:2009fw}.

Referring back to Chapter \ref{ch:introducing_the_method}, and the way the reconstruction was formulated in terms of matrices, the only modifications we have to make in order to reformulate the problem as a mass-squared-difference fit are the following: Define a vector $\mathbf{M}$ of mass-squared differences

\begin{align}
	\mathbf{M} = (M_1, M_2, M_3),
\end{align}
where
\begin{align}
	M_1 = M_D^2 - M_C^2, \, M_2 = M_C^2 - M_B^2, \, M_3 = M_B^2 - M_A^2,
\end{align}
and observe that the vector $\mathbf{S}$ may still be written as
\begin{align}
	\mathbf{S} = \mathbf{B}\mathbf{M} + \mathbf{C},
\end{align}
provided we let
\begin{align}
	\mathbf{B} = \begin{pmatrix}
					-1 & 0 & 0 \\
					0 & -1 & 0 \\
					0 & 0 & -1 \\
					0 & 0 & 0 \\
					-1 & 0 & 0  \\
					0 & -1 & 0  \\
					0 & 0 & -1  \\
					0 & 0 & 0 \\
	\end{pmatrix}.
\end{align}
Thus the reconstructed LSP momenta $\mathbf{P} = (p_A^x, p_A^y, p_A^z, E_A, p_{A'}^x, p_{A'}^y, p_{A'}^z, E_{A'})$ are still given as 
\begin{align}
	\mathbf{P} = \mathbf{A}^{-1}\mathbf{B}\mathbf{M} + \mathbf{A}^{-1}\mathbf{C},
\end{align}
where $\mathbf{M}$ and $\mathbf{B}$ are modified and $\mathbf{A}$ and $\mathbf{C}$ are as before.

This means that we can reformulate our problem to fit $M_{1,2,3}$ instead. However, since we in this case don't fit the masses themselves, our $\xi^2$ function,
\begin{align}
	\xi^2(\mathbf{M}) = \sum_n \left[(\hat p_{A}^2)_n - \frac{M_A^2}{M_\mathrm{norm}^2}\right]^2 + \left[(\hat p_{A'}^2)_n - \frac{M_{A'}^2}{M_\mathrm{norm}^2}\right]^2,\label{eq:xisquared_modified_repeat}
\end{align} 
has an unknown variable $M_A$. We choose to use the dilepton mass edge constraint, eq.\ \eqref{eq:MLSP_dilepton_edge}, to calculate the value of $M_A^2$ from the squared mass difference at each function evaluation. In terms of the mass-squared differences $M_{1,2,3}$, $M_A^2$ is given as
\begin{align}
	M_A^2 = M_3\left( \frac{M_2}{(m_{ll}^\mathrm{max})^ 2} - 1 \right).
\end{align}
We note that with these modifications, we have introduced another constraining equation into our problem, thus reducing the number of free parameters from two to one, as discussed in Chapter \ref{ch:introducing_the_method}. In addition, the minimization problem has been reduced from a four-dimensional one to a three-dimensional one. 

\begin{figure}[hbt]
	\centering
	\begin{subfigure}[b]{0.45\textwidth}
		\includegraphics[width=\textwidth]{figures/improving_combinatorics/herwigpp-MD-dileptonedge-fit-nocomb-nosmear-nocut.pdf} 
		\caption{ }
		\label{fig:MD_nocomb-nosmear}
	\end{subfigure}
	\begin{subfigure}[b]{0.45\textwidth}
		\includegraphics[width=\textwidth]{figures/improving_combinatorics/herwigpp-MD-dileptonedge-fit-nocomb-5psmear-nocut.pdf} 
		\caption{ }
		\label{fig:MD_nocomb-5psmear}
	\end{subfigure}
	\caption{Mass-difference minimizations on the Herwig++ dataset (a) without smearing and (b) with 5 \% momentum smearing, without combinatorics.}
\end{figure}
A fit of the unsmeared dataset with this method, not considering combinatorics, is shown in fig.\ \ref{fig:MD_nocomb-nosmear}. We have used the theoretical value of $m_{ll}^\mathrm{max} \approx 80 \mathrm{GeV}$ for the SPS1a masses, calculated using eq.\ \eqref{eq:invariant_mass_endpoint}. We have checked that the fit also in this case is independent of where we start the search. We also show the same fit on the dataset with 5 \% momentum smearing in fig.\ \ref{fig:MD_nocomb-5psmear}. In the last chapter we saw that with momentum smearing, the LSP mass was estimated to zero in many \marginpar{quantify} of the samples (fig. \ref{fig:moving_on-starting_point_sensitivity_no_combinatorics_5pmomsmear}) when we used the original formulation of the method. In this case all the samples have a nonzero LSP mass.
\begin{figure}[hbt]
	\centering
	\begin{subfigure}[b]{0.45\textwidth}
		\includegraphics[width=\textwidth]{figures/improving_combinatorics/herwigpp-MD-dileptonedge-fit-comb-nosmear-nocut_TMP.pdf} 
		\caption{ }
	\end{subfigure}
	\begin{subfigure}[b]{0.45\textwidth}
		\includegraphics[width=\textwidth]{figures/improving_combinatorics/herwigpp-MD-dileptonedge-fit-comb-nosmear-nocut_400-300-200-100.pdf}
		\caption{ } 
	\end{subfigure}

	\begin{subfigure}[b]{0.45\textwidth}
		\includegraphics[width=\textwidth]{figures/improving_combinatorics/herwigpp-MD-dileptonedge-fit-comb-nosmear-nocut_800-500-300-50.pdf} 
		\caption{ }
	\end{subfigure}
	\begin{subfigure}[b]{0.45\textwidth}
		\includegraphics[width=\textwidth]{figures/improving_combinatorics/herwigpp-MD-dileptonedge-fit-comb-nosmear-nocut_1000-100-80-30.pdf}
		\caption{ } 
	\end{subfigure}
	\caption{Mass-difference minimization on the unsmeared Herwig++ dataset with combinatorics done according to Webber, for the four different starting points used earlier.}
	\label{fig:MD_starting_point_sensitivity_combinatorics}
\end{figure}
\begin{figure}[hbt]
	\centering
	\begin{subfigure}[b]{0.45\textwidth}
		\includegraphics[width=\textwidth]{figures/improving_combinatorics/herwigpp-MD-dileptonedge-fit-comb-nosmear-cut100_TMP.pdf} 
		\caption{ }
	\end{subfigure}
	\begin{subfigure}[b]{0.45\textwidth}
		\includegraphics[width=\textwidth]{figures/improving_combinatorics/herwigpp-MD-dileptonedge-fit-comb-nosmear-cut100_400-300-200-100.pdf}
		\caption{ } 
	\end{subfigure}

	\begin{subfigure}[b]{0.45\textwidth}
		\includegraphics[width=\textwidth]{figures/improving_combinatorics/herwigpp-MD-dileptonedge-fit-comb-nosmear-cut100_800-500-300-50.pdf} 
		\caption{ }
	\end{subfigure}
	\begin{subfigure}[b]{0.45\textwidth}
		\includegraphics[width=\textwidth]{figures/improving_combinatorics/herwigpp-MD-dileptonedge-fit-comb-nosmear-cut100_1000-100-80-30.pdf}
		\caption{ } 
	\end{subfigure}
	\caption{Mass-difference minimization on the unsmeared Herwig++ dataset with combinatorics done according to Webber, for the four different starting points used earlier, subject to a $\xi^2$ cut of 100.}
	\label{fig:MD_starting_point_sensitivity_combinatorics_cut}
\end{figure}

We immediately investigate whether the modifications have affected the combinatorical problems we faced when using the original formulation, where we pick the lowest among all combinations for each event in each point. We saw in section \ref{sec:SP-dependence_webber} that the results were dependent on where the minimization search was started. In fig. \ref{fig:MD_starting_point_sensitivity_combinatorics} we show plots of the best-fit points using the same method for handling the combinatorical ambiguities, but the mass-squared-difference formulation of the $\xi^2$, minimized from the four different starting points used earlier. In fig.\ \ref{fig:MD_starting_point_sensitivity_combinatorics_cut} we show the same plots with the standard 100 GeV cut applied on the $\xi^2$. The differences between the plots appear less significant than in fig.\ \ref{fig:starting_point_sensitivity_combinatorics}. This is reflected in the mean values, which for the no-cut case range from 507 to 519 GeV in the four plots, and in the standard errors, which all are about 60 GeV. We also see that in fig.\ \ref{fig:MD_starting_point_sensitivity_combinatorics_cut}, where the cut is applied, the best-fit points surviving the cut lie in the same region in all four plots. The four different starting points are also consistent with respect to how many samples obtain convergence in the minimization (about 50 of 100), how many samples survive the $\xi^2$ cut (about 50 \% of the convergent samples), and how many events obtain the correct combinatorical choice in the best-fit point (about 25 \% of the convergent samples). \marginpar{Note: Check whether the remaining 75 \% of events are evenly divided between the other three combinations with same lepton-quark pairing!}

Statistically, the four plots in fig.\ \ref{fig:MD_starting_point_sensitivity_combinatorics} and \ref{fig:MD_starting_point_sensitivity_combinatorics_cut} are similar. But the individual event samples are not consistently fitted to the same points, and the same samples do not obtain convergence in all cases. In the case where no cut is applied, 32 of the 100 event samples obtain convergence from all four starting points, although each starting point has about 50 convergent samples. Only three of the samples obtain best-fit points less than $0.1$ GeV apart in all four masses in all four cases. The standard error between the four best-fit points for each event sample has an average of between 10 and 15 GeV for the four masses.\marginpar{Does this sentence make sense? Also, do the same analysis for the case with xisquared-cut.}



\section{Summing the combinations}
\label{sec:combinatorics-sum_all_contributions}
We have seen that when we minimize only the true combinatorical choice, the minimization is independent of starting point. If we include all combinations by always selecting the lowest among the values in each point, a starting-point dependence is introduced. The mathematical difference between these two problems is that in the former case, the $\xi^2$ is a smooth polynomial, while in the latter case it is not smooth. We can make the $\xi^2$ surface smooth if we add the different combinatorical choices together instead of choosing between them. Webber mentions this option in his article, but discards it, saying {\it ``The surface would be smooth if one added the $\xi^2$ contributions of all combinations,  but then the sensitivity to the correct solution is reduced and biases are introduced by the huge contributions of wrong combinations''} \cite{Webber:2009vm}. However, not all combinations will contribute equally if they are included. In \cite{Gripaios:2011jm} it is pointed out that some of the wrong combinations will tend to give reconstructions close to the true combinations. \marginpar{Discuss some more? What is the argument? What do they actually say?}

For the case where the dileptons differ in generation between the chains, which is what we have used in our analysis thus far, there are eight combinatorical possibilities. These can be divided into two categories depending on which lepton pair is paired with which quark. For each of these pairings, there are four combinations of near and far leptons. The matrix $\mathbf{A}$ is invariant, up to a permutation of rows, for each of the two categories. If we add all four combinations of near and far leptons for each event to the $\xi^2$, then the problem reduces from an eight-fold ambiguity to a two-fold ambiguity. 
\begin{figure}[hbt]
	\centering
	\includegraphics[width=0.8\textwidth]{figures/improving_combinatorics/herwigpp-4combosum-fit-nocomb-nosmear-nocut.pdf} 
	\caption{Minimization of the unsmeared {\tt Herwig++} dataset where all orderings of the leptons within the same chains are included.}
	\label{fig:4combosum_nocomb-nosmear}
\end{figure}
In fig.\ \ref{fig:4combosum_nocomb-nosmear} we show a fit of the unsmeared dataset where the $\xi^2$ has been constructed in this way, but where only the true pairing of quarks and leptons is considered. The mean values and standard errors for the mass fits in the 100 samples is $565 \pm 41\, \mathrm{GeV}, 180 \pm 34\, \mathrm{GeV}, 141 \pm 34\, \mathrm{GeV}$ and $98 \pm 35\, \mathrm{GeV}$, respectively. The standard error is quite large, about 40 GeV for all the four masses, but the mean values are very close to the true values in all four cases -- there is no bias. The minimization is also completely robust against starting points: For the four different starting points used earlier, all 100 bins obtain convergence in all cases, and each sample is fitted to the same point in all cases. \marginpar{The number of samples that converge is an important factor, 100 \% here vs 50 \% in the webber-jumping.}

\begin{figure}[hbt]
	\centering
	\includegraphics[width=0.8\textwidth]{figures/improving_combinatorics/herwigpp-8combosum-fit-nosmear-nocut.pdf} 	
	\caption{Minimization of the unsmeared {\tt Herwig++} dataset where all eight combinations are included.}
	\label{fig:8combosum-nosmear}
\end{figure}
If we include all eight combinations instead of only the four closest, then the results worsen considerably. This is shown in fig.\ \ref{fig:8combosum-nosmear}. While the robustness against starting points is retained, only half of the samples obtain convergence, and the errors on the masses are about 90 GeV for the three lightest particles and 150 GeV for the squark. There is a significant downward bias on the squark mass.

We proceed with investigating the method of summing only the four closest combinations. In this formulation, there are several options for handling the remaining two-fold combinatorical ambiguity. One option is to utilise the original method of ``jumping'' between values, always selecting the lowest among the combinations for each event at each mass point. When there are just two different values to choose between, the amount of jumping and the resulting difficulties might be reduced. We check this by starting the minimization from the four different starting points. We find in this case for the unsmeared dataset that $\sim 90$ samples converge in each case, while 84 of the samples converge in all four cases. 62 of the samples are fitted consistently, meaning that all four minimization agree on all four mass values within 0.1 GeV. The mean standard error on the best-fit points from the four different minimizations in the 84 samples is 9 GeV for the squark and about 4 GeV for the other three masses. We also find that about 80 \% of the events obtain the minimum with the correct combination, while the remaining 20 \% apparently obtain smaller values using the wrong combination of quarks and leptons.
\begin{figure}[hbt]
	\centering
	\begin{subfigure}[b]{0.45\textwidth}
		\includegraphics[width=\textwidth]{figures/improving_combinatorics/herwigpp-4combosum-fit-jump_comb-nosmear-nocut-TMP.pdf} 
		\caption{ }
	\end{subfigure}
	\begin{subfigure}[b]{0.45\textwidth}
		\includegraphics[width=\textwidth]{figures/improving_combinatorics/herwigpp-4combosum-fit-jump_comb-nosmear-nocut-400-300-200-100.pdf}
		\caption{ } 
	\end{subfigure}

	\begin{subfigure}[b]{0.45\textwidth}
		\includegraphics[width=\textwidth]{figures/improving_combinatorics/herwigpp-4combosum-fit-jump_comb-nosmear-nocut-800-500-300-50.pdf} 
		\caption{ }
	\end{subfigure}
	\begin{subfigure}[b]{0.45\textwidth}
		\includegraphics[width=\textwidth]{figures/improving_combinatorics/herwigpp-4combosum-fit-jump_comb-nosmear-nocut-1000-100-80-30.pdf}
		\caption{ } 
	\end{subfigure}
	\caption{Mass-difference minimization on the unsmeared Herwig++ dataset with combinatorics handled by summing the four closest combinations and jumping between the two quark-lepton combinations, for the four different starting points used earlier.}
	\label{fig:4combosum_starting_point_sensitivity_combinatorics-jumping}
\end{figure}

Another possibility is to combine a few events at a time, and minimize the $\xi^2$, with the lepton-combinations summed, separately for each of the quark-lepton combinations ({\it e.g.}\ each matrix $\mathbf{A}$) in each event. For instance, if we select two events, then there are four ways to combine the different $\mathbf{A}$ matrices among the events. If the lowest $\xi^2$ minimum value among these four tend to belong to the true $\mathbf{A}$ matrices, then this may be used to select the correct quark-lepton combinations before all events are summed together for the total mass fit. \marginpar{A BIG IF: Test if this actually works.}





\begin{itemize}
\item Smooth minimization, finds same minimum always. Include plot showing this?
\item Compare mean value and error to the original hopes of Webber, and to e.g. Cheng. 
\item However, biased. Cite cheng, wrong combinations bias downward. Desireable to reduce wrong combinations also within same matrix.
\item Investigate determinant.
\item Investigate SFL events. What happens to the minima if there are four different A matrices? Can we select the true combo among the four? Remember to also utilise the dilepton edge to discriminate.
\item Possible way out of combinatorics: Sum all contributions with same matrix A. Select between the two matrices in some way, either by Webber-jumping or by selecting on samples of fewer events. That actually works better than I thought. At LEAST something to include plots of. Note: Bias toward lower values, even after cut. Can be corrected by MC? 
\item A point for conclusions: Webber cites the Cheng papers in the original article, pointing out that his method is ``closest in spirit to those''. So maybe we can sum up by comparing, and saying something about that the Cheng version appears to have prospects for a more accurate determination?
\end{itemize}




















\clearpage
\section{The pile}



We saw in the previous chapter that the minimization was not well-defined, since it depended heavily on the choice of starting point. This was however a minimization done with the full inclusion of combinatorical ambiguities (see section \ref{sec:combinatorics}). While these ambiguities are clearly unavoidable -- we do not know \`{a} priori which particles belong together -- there may be other ways to handle them than the intuitive method applied in the above analysis. It might be possible to use other kinematical constraints, like the end-point method mentioned in section \ref{ch:introducing_the_method}, to rule out combinations. We could also try to minimize a separate $\xi^2$ for each combination in each event, and select the combination which gives the lowest value as the true one -- {\bf if} this turns out to correlate with the correct combination. In \cite{Gripaios:2011jm} there is a comprehensive discussion of the relationship between different combinations -- maybe we can exploit these relationships to pin down the correct combination? Or borrow some other ideas from it?

We should also investigate methods to identify events which contribute badly to the fit, so they can be dropped. We might try to use the event-wise $\xi^2$ for this, dropping events with a large minimum. We could also hope to use the determinant of the matrix A to identify events that are difficult to invert (small determinant), surmising that they also will give poor reconstruction. We should also investigate the determinant for the different combinatorical choices -- maybe we can rule out some combinations based on this? (Although several combinations share the same A.)

Below it turns out that the method seems insensitive to the LSP mass -- I am starting to think that the way they do it in \cite{Nojiri:2010dk}, fitting the squared mass differences instead, is all we can hope to do. But this should be tried together with using the dilepton end-point to pin the LSP mass, which MAY give a full reconstruction again. Or at least some more plots.




\begin{itemize}
	\item Minimize separately for each event? Do something to make it steep enough?
	\item combinatorical elimination: momentum cuts, angles, invariant mass edge. Angular selection: squarks produced more or less at rest (except if they come from gluino in a large mass-gap scenario). But if there is a mass gap further down, does this give anisotropies in angular distributions of opposite-chain particles? Don't focus on angles.
	\item migrad or similar for minimization of 1-event bins? Provide gradient, analytical or numerical?
	\item combinatorical issues with three or four hard squarks -- gluino-gluino vs gluino-squark vs squark-squark/squark-antisquark. Nllfast to give xsec. 
	\item Examine correlation between det(A) and minimization performance. 
	\item Include plots of the $\xi^2$ surface somewhere suitable -- maybe we can even compare combinatorics vs no combinatorics in a nice way?
\end{itemize}















































% %%%%%%%%%%%%%%%%%%%%%%%%%%%%%%%%%%%%%%%%
% \chapter{Improvements of the method}
% %%%%%%%%%%%%%%%%%%%%%%%%%%%%%%%%%%%%%%%
% To think about: How do we scan when we don't know true values? The fit is sensitive to initial values. Might it be fruitful to run many minimizations with different starting positions? That would essentially stack another 4-dimensional minimization on top of this one. Would scale CPU time enormously. Probably not that clever.

% What about discarding events with large minima? Do they seem to ruin the fit or is the minimal value not that good a measurement? Webber does this. 

% I have not looked much into using Det(A) as a measure of invertibility. It seems like invertibility is not really the big issue, but it might be worth investigating never the less. Maybe discard events based on Det(A)?

% One very unchewed idea: How about fitting the more fundamental SUSY soft parameters instead of the physical masses? I.e. substitute the soft masses instead of the physical ones? Can we gain something under different model assumptions, extract some co-varying parameters or something?






















\appendix

\chapter{A C++ implementation of the Nelder-Mead Simplex algorithm}
\label{ch:simplex}

The function to call for minimization is {\tt amoeba}. It returns true or false depending on whether convergence has been obtained within the set number of iterations.

\lstset{language=C++}
\begin{lstlisting}
	// Implementation of Nelder-Mead Simplex method:
double * alloc_vector(int cols)
{
	return (double *) malloc(sizeof(double) * cols);
}
void free_vector(double * vector , int cols)
{
	free(vector);
}
double ** alloc_matrix(int rows, int cols)
{
	int	i;
	double ** matrix = (double **) malloc(sizeof(double *) * rows);
	for (i = 0; i < rows; i++)
		matrix[i] = alloc_vector(cols);
	return matrix;
}
void free_matrix(double ** matrix, int rows, int cols)
{
	int	i;
	for (i =0; i < rows; i++)
		free_vector(matrix[i], cols);
	free(matrix);
}
double ** make_simplex(double * point, int dim)
{
	int i, j;
	double ** simplex = alloc_matrix(dim + 1, dim);
	for (i = 0; i < dim + 1; i++)
		for (j = 0; j < dim; j++)
			simplex[i][j] = point[j];
	for (i = 0; i < dim; i++)
		// simplex[i][i] += 1.0;
		simplex[i][i] *= 1.1;
	return simplex;
}
void evaluate_simplex(double ** simplex, int dim,double * fx,  double (*func)(double *, int, int, double, bool, vector<bool> &, vector<vector<mat>> &, vector<vector<vec>> &, vector<bool> &),
	int Nevents, int jBin, double Mnorm, bool combinatorics, vector<bool> &all_leptons_equal_list, vector<vector<mat>> &D_lists, vector<vector<vec>> &E_lists, vector<bool> &correct_combinatorics)
{
	int i;
	for (i = 0; i < dim + 1; i++)
	{
		correct_combinatorics.clear();
		fx[i] = (*func)(simplex[i], Nevents, jBin, Mnorm, combinatorics, all_leptons_equal_list, D_lists, E_lists, correct_combinatorics);
	}
}

void simplex_extremes(double *fx, int dim, int & ihi, int & ilo,int & inhi)
{
	int i;
	if (fx[0] > fx[1])
	{ ihi = 0; ilo = inhi = 1; }
	else
	{ ihi = 1; ilo = inhi = 0; }
	for (i = 2; i < dim + 1; i++)
		if (fx[i] <= fx[ilo])
			ilo = i;
		else if (fx[i] > fx[ihi])
			{ inhi = ihi; ihi = i; }
		else if (fx[i] > fx[inhi])
			inhi = i;
}
void simplex_bearings(double ** simplex, int dim,double * midpoint, double * line, int ihi)
{
	int i, j;
	for (j = 0; j < dim; j++)
		midpoint[j] = 0.0;
	for (i = 0; i < dim + 1; i++)
		if (i != ihi)
			for (j = 0; j < dim; j++)
				midpoint[j] += simplex[i][j];
	
	for (j = 0; j < dim; j++)
	{
		midpoint[j] /= dim;
		line[j] = simplex[ihi][j] - midpoint[j];
	}
}
int update_simplex(double * point, int dim, double & fmax,double * midpoint, double * line, double scale, double (*func)(double *, int, int, double, bool, vector<bool> &, vector<vector<mat>> &, vector<vector<vec>> &, vector<bool> &),
	int Nevents, int jBin, double Mnorm, bool combinatorics, vector<bool> &all_leptons_equal_list, vector<vector<mat>> &D_lists, vector<vector<vec>> &E_lists, vector<bool> &correct_combinatorics)
{
	int i, update =	0; 
	double * next = alloc_vector(dim), fx;
	for (i = 0; i < dim; i++)
		next[i] = midpoint[i] + scale * line[i];
	correct_combinatorics.clear();
	fx = (*func)(next, Nevents, jBin, Mnorm, combinatorics, all_leptons_equal_list, D_lists, E_lists, correct_combinatorics);
	if (fx < fmax)
	{
		for (i = 0; i < dim; i++)	
			point[i] = next[i];
		fmax = fx;
		update = 1;
	}
	free_vector(next, dim);
	return update;
}

void contract_simplex(double ** simplex, int dim, double * fx, int ilo, double (*func)(double *, int, int, double, bool, vector<bool> &, vector<vector<mat>> &, vector<vector<vec>> &, vector<bool> &),	int Nevents, int jBin, double Mnorm, bool combinatorics, vector<bool> &all_leptons_equal_list, vector<vector<mat>> &D_lists, vector<vector<vec>> &E_lists, vector<bool> &correct_combinatorics)
{
	int i, j;
	for (i = 0; i < dim + 1; i++)
		if (i != ilo)
		{
			for (j = 0; j < dim; j++)
				simplex[i][j] = (simplex[ilo][j]+simplex[i][j])*0.5;
			correct_combinatorics.clear();
			fx[i] = (*func)(simplex[i], Nevents, jBin, Mnorm, combinatorics, all_leptons_equal_list, D_lists, E_lists, correct_combinatorics);
		}
}


#define ZEPS 1e-30
int check_tol(double fmax, double fmin, double ftol)
{
double delta = fabs(fmax - fmin);
double accuracy = (fabs(fmax) + fabs(fmin)) * ftol;
// cout << delta << ", " << accuracy << ", " << ftol << endl;
return (delta < (accuracy + ZEPS));
}

bool amoeba(double *point, double &fmin, double (*func)(double *, int, int, double, bool, vector<bool> &, vector<vector<mat>> &, vector<vector<vec>> &, vector<bool> &), 
	double tol, int maxiter,
	int Nevents, int jBin, double Mnorm, bool combinatorics, vector<bool> &all_leptons_equal_list, vector<vector<mat>> &D_lists, vector<vector<vec>> &E_lists, vector<bool> &correct_combinatorics)
{
	// Usage: Point is an allocated dim-dimensional array of doubles
	// to be filled with coordinates of the best-fit point,
	// func is the function to minimize. 
	int dim = 3; // MODIFIED TO FIT MD
	int ihi, ilo, inhi, j;
	// double fmin;
	double * fx = alloc_vector(dim + 1);
	double * midpoint = alloc_vector(dim);
	double * line = alloc_vector(dim);
	double ** simplex = make_simplex(point, dim);
	evaluate_simplex(simplex, dim, fx, func, 
		Nevents, jBin, Mnorm, combinatorics, all_leptons_equal_list, D_lists, E_lists, correct_combinatorics);

	int iter = 0;
	while (iter < maxiter)
	{
		simplex_extremes(fx, dim, ihi, ilo, inhi);
		simplex_bearings(simplex, dim, midpoint, line, ihi);
		if (check_tol(fx[ihi], fx[ilo], tol)) { /*cout << "below tol = " << tol << endl;*/ break; }
		update_simplex(simplex[ihi], dim, fx[ihi],
		midpoint, line, -1.0, func, 
		Nevents, jBin, Mnorm, combinatorics, all_leptons_equal_list, D_lists, E_lists, correct_combinatorics);
		if (fx[ihi] < fx[ilo])
			update_simplex(simplex[ihi], dim, fx[ihi], midpoint, line, -2.0, func,
				Nevents, jBin, Mnorm, combinatorics, all_leptons_equal_list, D_lists, E_lists, correct_combinatorics);
		else if (fx[ihi] >= fx[inhi])
			if (!update_simplex(simplex[ihi], dim, fx[ihi], midpoint, line, 0.5, func, Nevents, jBin, Mnorm, combinatorics, all_leptons_equal_list, D_lists, E_lists, correct_combinatorics))
				contract_simplex(simplex, dim, fx, ilo, func, Nevents, jBin, Mnorm, combinatorics, all_leptons_equal_list, D_lists, E_lists, correct_combinatorics);
		iter += 1;
	}

	for (j = 0; j < dim; j++)
		point[j] = simplex[ilo][j];
	fmin = fx[ilo];
	free_vector(fx, dim);
	free_vector(midpoint, dim);
	free_vector(line, dim);
	free_matrix(simplex, dim + 1, dim);

	if (iter < maxiter)
	{
		return true;
	}
	else
		return false;
}
\end{lstlisting}








%%%%%%%%%%%%%%%%%%%%%%%%%%%%%%%%%%%%%%%%%%%%%%%%%%%%%%%%%%%%%%%%%
\chapter{An algorithm for generating on-shell two-body decays}
%%%%%%%%%%%%%%%%%%%%%%%%%%%%%%%%%%%%%%%%%%%%%%%%%%%%%%%%%%%%%%%%%
\label{ch:decayalgorithm}

 The particles will go back-to-back in the rest frame of the decaying particle - the direction is drawn randomly from a uniform spherical distribution. This can be achieved in spherical coordinates by the assignments
\begin{align}
	U, V &= \text{uniform}(0,1)\nonumber \\
	\phi &= 2\pi U\\
	\theta &= \arccos(2V-1) \nonumber
\end{align}
The decay algorithm, in python notation, is as follows. Particle 1 of 4-momentum $p_1$ decays to particle 2 and 3 of 4-momentum $p_2$ and $p_3$.

\lstset{language=Python} 
\begin{lstlisting}
	# Calculating four-momenta of particle 2&3 going back-to-back from
	# decay of particle 1 in the frame where particle 1 has 4-mom P1
	#
	#
	# particle 1 = decaying particle
	# particle 2 & particle 3 = decay products
	# primed system is rest frame of particle 1, unprimed is lab frame
	# rotated system is at rest in lab system,
	# but rotated so particle one goes in +x direction
	p1 = P1[0,1:4]
	p1abs = np.sqrt( float( np.dot( p1 , np.transpose(p1) ) ) ) # 3-momentum 
																# of particle 1 in 
												      			# lab frame

	# == Kinematical decay in RF of particle 1 ==
	p2absprime = 1.0/(2*m1) * np.sqrt( (m1**2-m2**2-m3**2)**2- 4*m2**2*m3**2 ) # abs-val
	# of 3-momentum of particle 2/3 in RF of particle 1

	U, V = np.random.uniform(0,1,2) # random 
	phi = 2*pi*U 					# point picking 
	theta = np.arccos(2*V-1) 		# on a sphere

	# Calculate cartesian 3- and 4-momentum of particle 2&3
	p2prime = np.matrix([ p2absprime*np.sin(theta)*np.cos(phi) , 
						  p2absprime*np.sin(theta)*np.sin(phi) , 
						  p2absprime*np.cos(theta) ])
	p3prime = -p2prime
	E2prime = np.sqrt( p2absprime**2 + m2**2 )
	E3prime = np.sqrt( p2absprime**2 + m3**2 )
	P2prime = np.matrix([ E2prime , p2prime[0,0] , p2prime[0,1] , p2prime[0,2] ])
	P3prime = np.matrix([ E3prime , p3prime[0,0] , p3prime[0,1] , p3prime[0,2] ])

	# == Back-transform to lab frame ==

	# First check whether it is necessary to boost

	if p1abs > 1e-10:

		# Lorentz boost along x-direction to get to rotated lab frame
		# (lab frame moves in negative x direction)
	 	vlab = -p1abs/np.sqrt(p1abs**2 + m1**2) # velocity of particle 1 in lab frame
		gamma = 1/np.sqrt(1-vlab**2)

		P2rot = np.matrix([ gamma*(P2prime[0,0] - vlab*P2prime[0,1]) , 
				      gamma*(P2prime[0,1] - vlab*P2prime[0,0]) ,
				      P2prime[0,2] , P2prime[0,3] ])
		P3rot = np.matrix([ gamma*(P3prime[0,0] - vlab*P3prime[0,1]) , 
				      gamma*(P3prime[0,1] - vlab*P3prime[0,0]) ,
				      P3prime[0,2] , P3prime[0,3] ])

		# == Rotate back to lab frame ==

		# Calculate the unit vectors of the rotated system axes in terms of lab axes

		# The definition is that x axis is along p1.
		# For the other axes we must make a choice - y&z directions are undetermined,
		# only the yz plane is determined from x choice. But since we have drawn 
		# random angles and the yz plane is not boosted, the choice does not matter
		# as long as we are consistent from event to event.
		# So we pick two vectors orthogonal to p1 and do Gram-Schmidt orthogonalization:
		v1 = p1
		v2 = np.matrix([ p1[0,1] , -p1[0,0] , 0 ])
		v3 = np.matrix([ p1[0,2] , 0 , -p1[0,0] ])

		u1 = v1
		u2 = v2 - proj(v2,u1)
		u3 = v3 - proj(v3,u1) - proj(v3,u2)

		xrot = u1/np.linalg.norm(u1) if np.linalg.norm(u1) > 0 else np.matrix([0,0,1])
		yrot = u2/np.linalg.norm(u2) if np.linalg.norm(u2) > 0 else np.matrix([0,1,0])
		zrot = u3/np.linalg.norm(u3) if np.linalg.norm(u3) > 0 else np.matrix([1,0,0])

		# Form a matrix T which takes a vector in the lab basis to a vector 
		# in the rotated basis by
		T = np.concatenate( (xrot , yrot , zrot) , axis=0 )
		# What we need is to rotate from rotated basis to lab basis, so we need the inverse
		# - which is the transpose, since rotation matrices are orthogonal. 
		# Also, to ease calculation, we let T be the 3x3 submatrix of T4, setting the [0,0]
		#component of T4 to 1 to leave time component invariant under this spatial rotation
		T4 = np.matrix([[1,     0,     0,    0],
						[0,T[0,0],T[0,1],T[0,2]],
						[0,T[1,0],T[1,1],T[1,2]],
						[0,T[2,0],T[2,1],T[2,2]] ])

		P2 = T4.T*P2rot.T
		P3 = T4.T*P3rot.T
		P2 = P2.T
		P3 = P3.T

	# If it was unneccessary, i.e. decay happened in lab frame, then
	else:
		P2 = P2prime
		P3 = P3prime
\end{lstlisting}






\bibliographystyle{unsrt}
\bibliography{thesis-bibliography}

\end{document}